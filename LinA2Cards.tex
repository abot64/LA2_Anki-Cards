% -*- coding: utf-8-unix -*-
%%%%%%%%%%%%%%%%%%%%%%%%%%%%%%%%%%%%%%%%%%%%%%%%%%%%
% The first part of the header needs to be copied
%       into the note options in Anki.
%%%%%%%%%%%%%%%%%%%%%%%%%%%%%%%%%%%%%%%%%%%%%%%%%%%%

% layout in Anki:
\documentclass[11pt]{article}
\usepackage[a4paper]{geometry}
\geometry{paperwidth=.5\paperwidth,paperheight=100in,left=2em,right=2em,bottom=1em,top=2em}
\pagestyle{empty}
\setlength{\parindent}{0in}

% hyphenation:
\usepackage[ngerman]{babel}

% encoding:
\usepackage[T1]{fontenc}
\usepackage[utf8]{inputenc}
\usepackage{lmodern}

% packages:
\usepackage{
  % maths:
  amssymb,
  amsmath,
  mathtools,  % (corrects some bugs in ams-packages)
  bm,         % (bold maths symbols)
  calc,
  nicefrac,
  stmaryrd,   % (symbols)
  % microtype, don't use this:
  %   line breaks will differ between latex & pdflatex output
  eqparbox,
  paralist,
  xcolor,
  bbm         % mathbbm = math-blackboard-bold
}
\renewcommand{\cite}[1]{\par\bigskip\hfill{\color{gray}\tiny\(\to\) #1}}
% Diagrams:
\usepackage[all]{xy}

% Blackboard:
\newcommand{\CC}{\mathbb{C}}
\newcommand{\FF}{\mathbb{F}}
\newcommand{\NN}{\mathbb{N}}
\newcommand{\QQ}{\mathbb{Q}}
\newcommand{\RR}{\mathbb{R}}
\newcommand{\ZZ}{\mathbb{Z}}
\newcommand{\ve}{\underline}

% (Redefined) symobls:
\newcommand*{\abs}[1]{\left\vert#1\right\vert}
\newcommand*{\norm}[1]{\left\|#1\right\|}
\newcommand*{\scprod}[2]{\langle #1, #2\rangle} %Skalarprodukt
\renewcommand{\phi}{\varphi}
\renewcommand{\epsilon}{\varepsilon}
%\renewcommand{\setminus}{\!-\!}
\renewcommand{\leq}{\leqslant}
\renewcommand{\geq}{\geqslant}
\newcommand{\tsmash}{\wedge}
\newcommand{\twedge}{\vee}
\newcommand{\from}{\leftarrow}
\newcommand{\noloc}{:\!}
% Special maps:
\newcommand{\id}{\mathrm{id}}

% Linear algebra:
\renewcommand{\vec}[1]{\mathbf{#1}}
\newcommand{\Potenz}{\mathfrak P}
%\newcommand{\im}{\mathrm{im}}
\newcommand{\hull}[1]{\langle #1 \rangle}
\newcommand{\Lraum}{\mathcal L}

% Categories:
\DeclareMathOperator{\Ar}{Ar}
\usepackage{bm}
\newcommand{\cat}[1]{\bm{\mathsf{#1}}}
\newcommand{\iTop}{\underline{\smash{\mathsf{Top}}}}  % internal hom in Top
\newcommand{\iAut}{\underline{\smash{\mathsf{Aut}}}}
\newcommand{\Aut}{\mathrm{Aut}}
\newcommand{\Mor}{\mathrm{Mor}}
\newcommand{\Hom}{\mathrm{Hom}}
\DeclareMathOperator{\signum}{sgn}
\DeclareMathOperator{\rank}{rk}
\DeclareMathOperator{\im}{im}
\DeclareMathOperator{\Abb}{Abb}
\DeclareMathOperator{\Mat}{Mat}

\mathchardef\mhyphen="2D
\newcommand*{\factor}[2]{\left.\raisebox{.1em}{\ensuremath{#1}}\middle/\raisebox{-.1em}{\ensuremath{#2}}\right.}

% arrows:
\newcommand*{\openhookrightarrow}{\mathrel{\ooalign{$\hookrightarrow$\cr\hidewidth\hbox{$\circ\,\,$}\cr}}}
\newcommand*{\closedhookrightarrow}{\mathrel{\ooalign{$\hookrightarrow$\cr\hidewidth\raise0.1ex\hbox{$\rule{0.5pt}{1ex}\;\;$}\cr}}}
\newcommand*{\openhookleftarrow}{\mathrel{\ooalign{$\hookleftarrow$\cr\hidewidth\hbox{$\circ\;$}\cr}}}
\newcommand*{\closedhookleftarrow}{\mathrel{\ooalign{$\hookleftarrow$\cr\hidewidth\raise0.1ex\hbox{$\rule{0.5pt}{1ex}\,\;$}\cr}}}
\newcommand*{\leadsleadsto}{\mathrel{\substack{\leadsto\\[-1em]\leadsto}}}


% text positioning in maths:
\newcommand{\ctext}[2]{\text{\parbox{#1}{\centering #2}}}
\newcommand{\ltext}[2]{\text{\parbox{#1}{\raggedright#2}}}
\newcommand{\rtext}[2]{\text{\parbox{#1}{\raggedleft#2}}}
\newcommand{\cttext}[2]{\text{\parbox[t]{#1}{\centering #2}}}
\newcommand{\lttext}[2]{\text{\parbox[t]{#1}{\raggedright#2}}}
\newcommand{\rttext}[2]{\text{\parbox[t]{#1}{\raggedleft#2}}}
% Hide
\newcommand{\hide}[1]{\parbox{0cm}{\raisebox{-7pt}[0cm]{\dots}}\color{white}#1\color{black}}
\newcommand{\hint}[1]{{\color{lightgray}(#1)}}
\let\olddots\dots
\renewcommand{\dots}{\,\olddots\,}

% bold text sans serif
\DeclareTextFontCommand{\textbf}{\bfseries\sffamily}
\hyphenation{
  Vektor-raum
  Vektor-raums
  Unter-vektor-raum
  Unter-vektor-raums
  Ortho-normal-basis
  Skalar-produkt
  Standard-skalar-produkt
}
%%%%%%%%%%%%%%%%%%%%%%%%%%%%%%%%%%%%%%%%%%%%%%%%%%%%
% Following part of header NOT to be copied into
%            the note options in Anki.
%          ! Anki will throw an errow !
%%%%%%%%%%%%%%%%%%%%%%%%%%%%%%%%%%%%%%%%%%%%%%%%%%%%%
%
%  pdf layout:
%
\geometry{paperheight=74.25mm}
\usepackage{pgfpages}
\pagestyle{empty}
\pgfpagesuselayout{8 on 1}[a4paper,border shrink=0cm]
\makeatletter
\@tempcnta=1\relax
\loop\ifnum\@tempcnta<9\relax
\pgf@pset{\the\@tempcnta}{bordercode}{\pgfusepath{stroke}}
\advance\@tempcnta by 1\relax
\repeat
\makeatother
%
%  notes, fields, tags:
%
\newcommand{\xfield}[1]{
        #1\par
        \vfill
        {\tiny\texttt{\parbox[t]{\textwidth}{\localtag\\\globaltag\hfill\uuid}}}
        \newpage}
\newenvironment{field}{}{\newpage}
\newif\ifnote
\newenvironment{note}{\notetrue}{\notefalse}
\newcommand{\localtag}{}
\newcommand{\globaltag}{}
\newcommand{\uuid}{}
\newcommand{\tags}[1]{
    \ifnote
        \renewcommand{\localtag}{#1}
    \else
        \renewcommand{\globaltag}{#1}
    \fi
    }
\newcommand{\xplain}[1]{\renewcommand{\uuid}{#1}}
%
%%%%%%%%%%%%%%%%%%%%%%%%%%%%%%%%%%%%%%%%%%%%%%%%%%%%
% The following line again needs to be copied
% into Anki:
\begin{document}
%%%%%%%%%%%%%%%%%%%%%%%%%%%%%%%%%%%%%%%%%%%%%%%%%%%%

\tags{LinA-II-10-Skalarprodukte}
%%%%%%%%% Vorlesung 1 %%%%%%%%%%%%
\begin{note}
    \xplain{6fc8a1b6-b98b-11ec-8422-0242ac120002}
    \tags{Def}
    \begin{field} %Frage 1
    Sei \(K\) ein Körper, \(V\) ein \(K\)-VR. Eine \textbf{Bilinearform} auf \(V\) ist eine (bilineare) Abbildung  \(\beta\colon V\times V \to K\), für die gilt:
    \begin{itemize}
        \item[(1)] \hide{\(\beta(\vec{v_1} + \vec{v_2}, \vec{w}) = \beta(\vec{v_1}, \vec{w}) + \beta(\vec{v_2}, \vec{w})\) für alle \(\vec{v_1, v_2, w} \in V\)}
        \item[(2)] \(\beta(s\cdot\vec{v},\vec{w}) = s\cdot\beta(\vec{v},\vec{w})\) für alle \(s\in K\),\(\vec{v},\vec{w}\in V\)
        \item[(3)]\(\beta(\vec{v}, \vec{w_1} + \vec{w_2}) = \beta(\vec{v}, \vec{w_1}) + \beta(\vec{v}, \vec{w_2})\) für alle \(\vec{v, w_1, w_2} \in V\)
        \item[(4)]\(\beta(\vec{v},s\cdot\vec{w}) = s\cdot\beta(\vec{v},\vec{w})\) für alle \(s\in K\),\(\vec{v},\vec{w}\in V\)
    \end{itemize}
    \end{field}
    \begin{field} %Antwort 1
    Sei \(K\) ein Körper, \(V\) ein \(K\)-VR. Eine \textbf{Bilinearform} auf \(V\) ist eine (bilineare) Abbildung  \(\beta\colon V\times V \to K\), für die gilt:
    \begin{itemize}
        \item[(1)] \(\beta(\vec{v_1} + \vec{v_2}, \vec{w}) = \beta(\vec{v_1}, \vec{w}) + \beta(\vec{v_2}, \vec{w})\) für alle \(\vec{v_1, v_2, w} \in V\)
        \item[(2)] \(\beta(s\cdot\vec{v},\vec{w}) = s\cdot\beta(\vec{v},\vec{w})\) für alle \(s\in K\),\(\vec{v},\vec{w}\in V\)
        \item[(3)]\(\beta(\vec{v}, \vec{w_1} + \vec{w_2}) = \beta(\vec{v}, \vec{w_1}) + \beta(\vec{v}, \vec{w_2})\) für alle \(\vec{v, w_1, w_2} \in V\)
        \item[(4)]\(\beta(\vec{v},s\cdot\vec{w}) = s\cdot\beta(\vec{v},\vec{w})\) für alle \(s\in K\),\(\vec{v},\vec{w}\in V\)
    \end{itemize}
    \cite{Def. ~10.1}
    \end{field}
    \begin{field} %Frage 2
    Sei \(K\) ein Körper, \(V\) ein \(K\)-VR. Eine \textbf{Bilinearform} auf \(V\) ist eine (bilineare) Abbildung  \(\beta\colon V\times V \to K\), für die gilt:
    \begin{itemize}
        \item[(1)] \(\beta(\vec{v_1} + \vec{v_2}, \vec{w}) = \beta(\vec{v_1}, \vec{w}) + \beta(\vec{v_2}, \vec{w})\) für alle \(\vec{v_1, v_2, w} \in V\)
        \item[(2)] \hide{\(\beta(s\cdot\vec{v},\vec{w}) = s\cdot\beta(\vec{v},\vec{w})\) für alle \(s\in K\),\(\vec{v},\vec{w}\in V\)}
        \item[(3)]\(\beta(\vec{v}, \vec{w_1} + \vec{w_2}) = \beta(\vec{v}, \vec{w_1}) + \beta(\vec{v}, \vec{w_2})\) für alle \(\vec{v, w_1, w_2} \in V\)
        \item[(4)]\(\beta(\vec{v},s\cdot\vec{w}) = s\cdot\beta(\vec{v},\vec{w})\) für alle \(s\in K\),\(\vec{v},\vec{w}\in V\)
    \end{itemize}
    \end{field}
    \begin{field} %Antwort 2
    Sei \(K\) ein Körper, \(V\) ein \(K\)-VR. Eine \textbf{Bilinearform} auf \(V\) ist eine (bilineare) Abbildung  \(\beta\colon V\times V \to K\), für die gilt:
    \begin{itemize}
        \item[(1)] \(\beta(\vec{v_1} + \vec{v_2}, \vec{w}) = \beta(\vec{v_1}, \vec{w}) + \beta(\vec{v_2}, \vec{w})\) für alle \(\vec{v_1, v_2, w} \in V\)
        \item[(2)] \(\beta(s\cdot\vec{v},\vec{w}) = s\cdot\beta(\vec{v},\vec{w})\) für alle \(s\in K\),\(\vec{v},\vec{w}\in V\)
        \item[(3)]\(\beta(\vec{v}, \vec{w_1} + \vec{w_2}) = \beta(\vec{v}, \vec{w_1}) + \beta(\vec{v}, \vec{w_2})\) für alle \(\vec{v, w_1, w_2} \in V\)
        \item[(4)]\(\beta(\vec{v},s\cdot\vec{w}) = s\cdot\beta(\vec{v},\vec{w})\) für alle \(s\in K\),\(\vec{v},\vec{w}\in V\)
    \end{itemize}
    \cite{Def. ~10.1}
    \end{field}
    \begin{field} %Frage 3
    Sei \(K\) ein Körper, \(V\) ein \(K\)-VR. Eine \textbf{Bilinearform} auf \(V\) ist eine (bilineare) Abbildung  \(\beta\colon V\times V \to K\), für die gilt:
    \begin{itemize}
        \item[(1)] \(\beta(\vec{v_1} + \vec{v_2}, \vec{w}) = \beta(\vec{v_1}, \vec{w}) + \beta(\vec{v_2}, \vec{w})\) für alle \(\vec{v_1, v_2, w} \in V\)
        \item[(2)] \(\beta(s\cdot\vec{v},\vec{w}) = s\cdot\beta(\vec{v},\vec{w})\) für alle \(s\in K\),\(\vec{v},\vec{w}\in V\)
        \item[(3)]\hide{\(\beta(\vec{v}, \vec{w_1} + \vec{w_2}) = \beta(\vec{v}, \vec{w_1}) + \beta(\vec{v}, \vec{w_2})\) für alle \(\vec{v, w_1, w_2} \in V\)}
        \item[(4)]\(\beta(\vec{v},s\cdot\vec{w}) = s\cdot\beta(\vec{v},\vec{w})\) für alle \(s\in K\),\(\vec{v},\vec{w}\in V\)
    \end{itemize}
    \end{field}
    \begin{field} %Antwort 3
    Sei \(K\) ein Körper, \(V\) ein \(K\)-VR. Eine \textbf{Bilinearform} auf \(V\) ist eine (bilineare) Abbildung  \(\beta\colon V\times V \to K\), für die gilt:
    \begin{itemize}
        \item[(1)] \(\beta(\vec{v_1} + \vec{v_2}, \vec{w}) = \beta(\vec{v_1}, \vec{w}) + \beta(\vec{v_2}, \vec{w})\) für alle \(\vec{v_1, v_2, w} \in V\)
        \item[(2)] \(\beta(s\cdot\vec{v},\vec{w}) = s\cdot\beta(\vec{v},\vec{w})\) für alle \(s\in K\),\(\vec{v},\vec{w}\in V\)
        \item[(3)]\(\beta(\vec{v}, \vec{w_1} + \vec{w_2}) = \beta(\vec{v}, \vec{w_1}) + \beta(\vec{v}, \vec{w_2})\) für alle \(\vec{v, w_1, w_2} \in V\)
        \item[(4)]\(\beta(\vec{v},s\cdot\vec{w}) = s\cdot\beta(\vec{v},\vec{w})\) für alle \(s\in K\),\(\vec{v},\vec{w}\in V\)
    \end{itemize}
    \cite{Def. ~10.1}
    \end{field}
    \begin{field} %Frage 4
    Sei \(K\) ein Körper, \(V\) ein \(K\)-VR. Eine \textbf{Bilinearform} auf \(V\) ist eine (bilineare) Abbildung  \(\beta\colon V\times V \to K\), für die gilt:
    \begin{itemize}
        \item[(1)] \(\beta(\vec{v_1} + \vec{v_2}, \vec{w}) = \beta(\vec{v_1}, \vec{w}) + \beta(\vec{v_2}, \vec{w})\) für alle \(\vec{v_1, v_2, w} \in V\)
        \item[(2)] \(\beta(s\cdot\vec{v},\vec{w}) = s\cdot\beta(\vec{v},\vec{w})\) für alle \(s\in K\),\(\vec{v},\vec{w}\in V\)
        \item[(3)]\(\beta(\vec{v}, \vec{w_1} + \vec{w_2}) = \beta(\vec{v}, \vec{w_1}) + \beta(\vec{v}, \vec{w_2})\) für alle \(\vec{v, w_1, w_2} \in V\)
        \item[(4)]\hide{\(\beta(\vec{v},s\cdot\vec{w}) = s\cdot\beta(\vec{v},\vec{w})\) für alle \(s\in K\),\(\vec{v},\vec{w}\in V\)}
    \end{itemize}
    \end{field}
    \begin{field}%Antwort 4
    Sei \(K\) ein Körper, \(V\) ein \(K\)-VR. Eine \textbf{Bilinearform} auf \(V\) ist eine (bilineare) Abbildung  \(\beta\colon V\times V \to K\), für die gilt:
    \begin{itemize}
        \item[(1)] \(\beta(\vec{v_1} + \vec{v_2}, \vec{w}) = \beta(\vec{v_1}, \vec{w}) + \beta(\vec{v_2}, \vec{w})\) für alle \(\vec{v_1, v_2, w} \in V\)
        \item[(2)] \(\beta(s\cdot\vec{v},\vec{w}) = s\cdot\beta(\vec{v},\vec{w})\) für alle \(s\in K\),\(\vec{v},\vec{w}\in V\)
        \item[(3)]\(\beta(\vec{v}, \vec{w_1} + \vec{w_2}) = \beta(\vec{v}, \vec{w_1}) + \beta(\vec{v}, \vec{w_2})\) für alle \(\vec{v, w_1, w_2} \in V\)
        \item[(4)]\(\beta(\vec{v},s\cdot\vec{w}) = s\cdot\beta(\vec{v},\vec{w})\) für alle \(s\in K\),\(\vec{v},\vec{w}\in V\)
    \end{itemize}
    \cite{Def. ~10.1}
    \end{field}
\end{note}

\begin{note}
    \tags{Satz}
    \xplain{d607e83e-b98f-11ec-8422-0242ac120002}
    \begin{field}
    Jede Bilinearform \(\beta\) auf \(K^n\) liefert eine Matrix \hide{\(M(\beta)\in \Mat_K(n\times n)\) der Gestalt}
    \end{field}
    \begin{field}
    Jede Bilinearform \(\beta\) auf \(K^n\) liefert eine Matrix \(M(\beta)\in \Mat_K(n\times n)\) der Gestalt
    \[M(\beta) := \beta(\vec{e}_i, \vec{e}_j)_{ij}\] \cite{Satz ~10.2}
    \end{field}
    \begin{field}
    Jede Matrix \(A\in\Mat_K(n\times n)\) liefert eine \hide{Bilinearform auf \(K^n\)} wie folgt:
    \end{field}
    \begin{field}
    Jede Matrix \(A\in\Mat_K(n\times n)\) liefert eine Bilinearform auf \(K^n\) wie folgt:
    \begin{align*}
    \beta_A: K^n\times K^n &\longrightarrow K \\
    (\vec{v}, \vec{w}) &\mapsto \vec{v}^T A \vec{w}
    \end{align*}
    \cite{Satz ~10.2}
    \end{field}
    \begin{field}
    Die Menge der Bilinearformen auf \(K^n\) und die Menge der \(n\times n\) Matrizen über \(K\) sind \hide{isomorph}.
    \end{field}
    \begin{field}
    Die Menge der Bilinearformen auf \(K^n\) und die Menge der \(n\times n\) Matrizen über \(K\) sind isomorph.
    \cite{Satz ~10.2}
    \end{field}
\end{note}

\begin{note}
    \tags{Def}
    \xplain{51387522-ba4a-11ec-8422-0242ac120002}
    \begin{field}
    Die \textbf{darstellende Matrix} einer Bilinearform \(\beta\) bezüglich einer Basis \(B=(\vec{b}_i)_i\) ist gegeben durch \hide{\[M_B(\beta) := \beta(\vec{b}_i, \vec{b}_j)_{ij}\]}
    \end{field}
    \begin{field}
    Die \textbf{darstellende Matrix} einer Bilinearform \(\beta\) bezüglich einer Basis \(B=(\vec{b}_i)_i\) ist gegeben durch
    \[M_B(\beta) := \beta(\vec{b}_i, \vec{b}_j)_{ij}\]
    \cite{Def. ~10.3}
    \end{field}
\end{note}

\begin{note}
    \tags{Def}
    \xplain{b9664264-bb12-11ec-8422-0242ac120002}
    \begin{field}
        Zwei Matrizen \(A, A'\) sind \textbf{kongruent}, falls es \hide{eine invertierbare Matrix \(S\) gibt mit}
    \end{field}

    \begin{field}
        Zwei quadratische Matrizen \(A, A'\) sind \textbf{kongruent}, falls es eine invertierbare Matrix \(S\) gibt mit
        \[ A' = S^TAS\]
    \cite{Def. ~10.5}
    \end{field}
\end{note}

%%%%%%%%% Vorlesung 2 %%%%%%%%%%%%
\begin{note}
    \tags{Def}
    \xplain{57381bea-bbe3-11ec-8422-0242ac120002}

    \xfield{
        Eine Bilinearform \(\beta\) auf \(V\) ist \textbf{symmetrisch}, falls \dots
    }
    \begin{field}
        Eine Bilinearform \(\beta\) auf \(V\) ist \textbf{symmetrisch}, falls für alle \(\vec{v},\vec{w}\in V\)
        \[\beta(\vec{v},\vec{w}) = \beta(\vec{w},\vec{v})\]
        \cite{Def. 10.7}
    \end{field}

    \xfield{
        Eine Bilinearform \(\beta\) auf \(V\) ist \textbf{schiefsymmetrisch}, falls \dots
    }
    \begin{field}
        Eine Bilinearform \(\beta\) auf \(V\) ist \textbf{schiefsymmetrisch}, falls für alle \(\vec{v},\vec{w}\in V\)
        \[\beta(\vec{v},\vec{w}) = - \beta(\vec{w},\vec{v})\]
        \cite{Def. 10.7}
    \end{field}

    \xfield{
        Eine Bilinearform \(\beta\) auf \(V\) ist \textbf{alternierend}, falls \dots
    }
    \begin{field}
        Eine Bilinearform \(\beta\) auf \(V\) ist \textbf{alternierend}, falls für alle \(\vec{v}\in V\)
        \[\beta(\vec{v},\vec{v}) = 0\]
        \cite{Def. 10.7}
    \end{field}
\end{note}

\begin{note}
    \tags{Satz}
    \xplain{fb0e2cdc-bbf2-11ec-8422-0242ac120002}

    \xfield{
        Eine Bilinearform \(\beta\) ist \textbf{symmetrisch} genau dann, wenn \dots
    \hint{darstellende Matrix}
    }
    \begin{field}
        Eine Bilinearform \(\beta\) ist \textbf{symmetrisch} genau dann, wenn ihre darstellende Matrix \(M(\beta)\) \textbf{symmetrisch} ist:
        \[M(\beta)^T = M(\beta)\]
        \cite{Satz 10.9}
    \end{field}

    \xfield{
    Eine Bilinearform \(\beta\) ist \textbf{schiefsymmetrisch} genau dann, wenn \dots
    \hint{darstellende Matrix}
    }
    \begin{field}
        Eine Bilinearform \(\beta\) ist \textbf{schiefsymmetrisch} genau dann, wenn ihre darstellende Matrix \(M(\beta)\) \textbf{schiefsymmetrisch} ist:
        \[M(\beta)^T = -M(\beta)\]
        \cite{Satz 10.9}
    \end{field}

    \xfield{
        Eine Bilinearform \(\beta\) ist \textbf{alternierend} genau dann, wenn \dots
    \hint{darstellende Matrix}
    }
    \begin{field}
        Eine Bilinearform \(\beta\) ist \textbf{alternierend} genau dann, wenn für ihre darstellende Matrix \(M(\beta)\) gilt:
        \[M(\beta)^T = -M(\beta)\]
        und
        \[M(\beta)_{ii} = 0 \text{ für alle }i\]
        \cite{Satz 10.9}
    \end{field}
\end{note}

\begin{note}
    \tags{Def}
    \xplain{7e2e222e-bbf4-11ec-8422-0242ac120002}

    \begin{field} %Lücke 1
        Eine \textbf{Sesquilinearform} \(\eta\) auf einem \(\CC\)-Vektorraum \(V\) ist eine Abbildung \(\eta\colon V\times V \longrightarrow \CC\) mit folgenden Eigenschaften:
        \begin{itemize}
            \item \hide{\(\eta\) ist linear in der ersten Koordinate:\\
                \(\eta(\vec{v}_1+s\vec{v}_2, \vec{w}) = \eta(\vec{v}_1, \vec{w}) + s\cdot \eta(\vec{v}_2,\vec{w})\)\\
                für alle \(\vec{v_1, v_2, w} \in V\) und alle \(s\in\CC\)}
            \item \(\eta\) ist semilinear in der zweiten Koordinate:\\
                \(\eta(\vec{v}, \vec{w}_1+\vec{w}_2) = \eta(\vec{v}, \vec{w}_1) + \eta(\vec{v},\vec{w}_2)\)\\
                \(\eta(\vec{v}, s\vec{w}) = \bar{s}\cdot \eta(\vec{v},\vec{w})\)\\
                für alle \(\vec{v, w, w_1, w_2} \in V\) und alle \(s\in\CC\)
        \end{itemize}
    \end{field}

    \begin{field} %Antwort 1
        Eine \textbf{Sesquilinearform} \(\eta\) auf einem \(\CC\)-Vektorraum \(V\) ist eine Abbildung \(\eta\colon V\times V \longrightarrow \CC\) mit folgenden Eigenschaften:
        \begin{itemize}
            \item \(\eta\) ist linear in der ersten Koordinate:\\
                \(\eta(\vec{v}_1+s\vec{v}_2, \vec{w}) = \eta(\vec{v}_1, \vec{w}) + s\cdot \eta(\vec{v}_2,\vec{w})\)\\
                für alle \(\vec{v_1, v_2, w} \in V\) und alle \(s\in\CC\)
            \item \(\eta\) ist semilinear in der zweiten Koordinate:\\
                \(\eta(\vec{v}, \vec{w}_1+\vec{w}_2) = \eta(\vec{v}, \vec{w}_1) + \eta(\vec{v},\vec{w}_2)\)\\
                \(eta(\vec{v}, s\vec{w}) = \bar{s}\cdot \eta(\vec{v},\vec{w})\)\\
                für alle \(\vec{v, w, w_1, w_2} \in V\) und alle \(s\in\CC\)
        \end{itemize}
        \cite{Def. 10.10}
    \end{field}

    \begin{field} %Lücke 2
        Eine \textbf{Sesquilinearform} \(\eta\) auf einem \(\CC\)-Vektorraum \(V\) ist eine Abbildung \(\eta\colon V\times V \longrightarrow \CC\) mit folgenden Eigenschaften:
        \begin{itemize}
            \item \(\eta\) ist linear in der ersten Koordinate:\\
                \(\eta(\vec{v}_1+s\vec{v}_2, \vec{w}) = \eta(\vec{v}_1, \vec{w}) + s\cdot \eta(\vec{v}_2,\vec{w})\)\\
                für alle \(\vec{v_1, v_2, w} \in V\) und alle \(s\in\CC\)
            \item \hide{\(\eta\) ist semilinear in der zweiten Koordinate:\\
                \(\eta(\vec{v}, \vec{w}_1+\vec{w}_2) = \eta(\vec{v}, \vec{w}_1) + \eta(\vec{v},\vec{w}_2)\)\\
                \(\eta(\vec{v}, s\vec{w}) = \bar{s}\cdot \eta(\vec{v},\vec{w})\)\\
                für alle \(\vec{v, w, w_1, w_2} \in V\) und alle \(s\in\CC\)}
        \end{itemize}
    \end{field}
    \begin{field} %Antwort
        Eine \textbf{Sesquilinearform} \(\eta\) auf einem \(\CC\)-Vektorraum \(V\) ist eine Abbildung \(\eta\colon V\times V \longrightarrow \CC\) mit folgenden Eigenschaften:
        \begin{itemize}
            \item \(\eta\) ist linear in der ersten Koordinate:\\
                \(\eta(\vec{v}_1+s\vec{v}_2, \vec{w}) = \eta(\vec{v}_1, \vec{w}) + s\cdot \eta(\vec{v}_2,\vec{w})\)\\
                für alle \(\vec{v_1, v_2, w} \in V\) und alle \(s\in\CC\)
            \item\(\eta\) ist semilinear in der zweiten Koordinate:\\
                \(\eta(\vec{v}, \vec{w}_1+\vec{w}_2) = \eta(\vec{v}, \vec{w}_1) + \eta(\vec{v},\vec{w}_2)\)\\
                \(\eta(\vec{v}, s\vec{w}) = \bar{s}\cdot \eta(\vec{v},\vec{w})\)\\
                für alle \(\vec{v, w, w_1, w_2} \in V\) und alle \(s\in\CC\)
        \end{itemize}
        \cite{Def. 10.10}
    \end{field}

    \xfield{Eine Sesquilinearform \(\eta\) ist \textbf{hermitesch}, falls \dots}
    \begin{field}
        Eine Sesquilinearform \(\eta\) ist \textbf{hermitesch}, falls gilt
        \[\eta(\vec{v},\vec{w}) = \overline{\eta(\vec{w},\vec{v})}\]
        für alle \(\vec{v},\vec{w}\in V\)
        \cite{Def. 10.10}
    \end{field}
\end{note}

\begin{note}
    \tags{Satz}
    \xplain{bd8dcf48-bc10-11ec-8422-0242ac120002}

    \xfield{
        Jede Sesquilinearform \(\eta\) liefert eine Matrix der Form \dots
    }
    \begin{field}
        Jede Sesquilinearform \(\eta\) liefert eine Matrix \(M(\eta) \in \Mat_{\CC}(n\times n)\) der Form
        \[M(\eta) := \eta(\vec{e}_i, \vec{e}_j)_{ij}\]
        \cite{Satz 10.11}
    \end{field}

    \xfield{
        Zu einer gegebenen komplexen Matrix \(A\) existiert eine Sesquilinearform \(\eta\) wie folgt:\\
        \dots
    }
    \begin{field}
        Zu einer gegebenen komplexen quadratischen Matrix \(A\) existiert eine Sesquilinearform \(\eta\) wie folgt:
        \[\eta_A(\vec{v},\vec{w})  := \vec{v}^T A \overline{\vec{w}}\]
        \cite{Satz 10.11}
    \end{field}
\end{note}

\begin{note}
    \tags{Def}
    \xplain{ca81504e-bc10-11ec-8422-0242ac120002}

    \xfield{Eine symmetrische Bilinearform \(\beta\) auf einem \\ \(\RR\)-Vektorraum ist \textbf{positiv definit}, falls \dots}
    \begin{field}
        Eine symmetrische Bilinearform \(\beta\) auf einem \\ \(\RR\)-Vektorraum ist \textbf{positiv definit}, falls
        \[\beta(\vec{v,v}) > 0 \text{ für alle }\vec{v}\in V\setminus\{\vec{0}\}\]
        \cite{Def. 10.14}
    \end{field}

    \xfield{Eine hermitesche Bilinearform \(\beta\) auf einem \(\CC\)-Vektorraum ist \textbf{positiv definit}, falls \dots}
    \begin{field}
        Eine hermitesche Bilinearform \(\beta\) auf einem \(\CC\)-Vektorraum ist \textbf{positiv definit}, falls
        \[\beta(\vec{v,v}) > 0 \text{ für alle }\vec{v}\in V\setminus\{\vec{0}\}\]
        \cite{Def. 10.14}
    \end{field}
\end{note}

\begin{note}
    \tags{Def}
    \xplain{d1b1125a-bc10-11ec-8422-0242ac120002}

    \xfield{
        Ein \textbf{Skalarprodukt} auf einem \(\RR\)-Vektorraum ist \dots
    }
    \begin{field}
        Ein \textbf{Skalarprodukt} auf einem \(\RR\)-Vektorraum ist eine positiv definite symmetrische Bilinearform.
        \cite{Def. 10.15}
    \end{field}

    \xfield{
        Ein \textbf{Skalarprodukt} auf einem \(\CC\)-Vektorraum ist \dots
    }
    \begin{field}
        Ein \textbf{Skalarprodukt} auf einem \(\CC\)-Vektorraum ist eine positiv definite hermitesche Bilinearform.
        \cite{Def. 10.15}
    \end{field}

    \xfield{
        Ein \textbf{euklidischer Vektorraum} ist \dots
    }
    \begin{field}
        Ein \textbf{euklidischer Vektorraum} \((V, \scprod{\cdot}{\cdot}) \) ist ein \\ \(\RR\)-Vektorraum \(V\) zusammen mit einem Skalarprodukt \(\scprod{\cdot}{\cdot}\).
        \cite{Def. 10.15}
    \end{field}

    \xfield{
        Ein \textbf{unitärer Vektorraum} ist \dots
    }
    \begin{field}
        Ein \textbf{unitärer Vektorraum} \((V, \scprod{\cdot}{\cdot}) \) ist ein \\ \(\CC\)-Vektorraum \(V\) zusammen mit einem Skalarprodukt \(\scprod{\cdot}{\cdot}\).
        \cite{Def. 10.15}
    \end{field}
\end{note}

\begin{note}
    \tags{Def}
    \xplain{d929b74e-bc10-11ec-8422-0242ac120002}
    \begin{field}
        Die assoziierte \textbf{Norm} zu einem euklidischen oder unitären Vektorraum \((V, \scprod{\cdot}{\cdot})\) ist gegeben durch
        \begin{align*}
            \norm{\cdot}\colon V \longrightarrow &\RR \\
            \vec{v} \mapsto &\phantom{\sqrt{\scprod{\vec{v}}{\vec{v}}}}
        \end{align*}
    \end{field}
    \begin{field}
        Die assoziierte \textbf{Norm} zu einem euklidischen oder unitären Vektorraum \((V, \scprod{\cdot}{\cdot})\) ist gegeben durch
        \begin{align*}
            \norm{\cdot}\colon &V \longrightarrow \RR \\
            &\vec{v} \mapsto \sqrt{\scprod{\vec{v}}{\vec{v}}}
        \end{align*}
        (Die Norm wird durch das Skalarprodukt \textbf{induziert}.)
        \cite{Def. 10.15}
    \end{field}
\end{note}

\begin{note}
    \tags{Satz}
    \xplain{e0e3bcb4-bc10-11ec-8422-0242ac120002}
    \begin{field}
        In jedem euklidischen oder unitären Vektorraum \((V, \scprod{\cdot}{\cdot})\) gilt:
        \begin{enumerate}[(i)]
            \item \hint{Verhältnis Norm und 0}\hide{\(\norm{\vec{v}} \geq 0\) für alle \(\vec{v}\in V\) \\
                  \(\norm{\vec{v}} = 0 \Leftrightarrow \vec{v} = \vec{0}\)}
            \item \(\norm{s\cdot\vec{v}} = \abs{s}\norm{\vec{v}}\)
            \item \textbf{Dreiecksungleichung}:\\
                  \(\norm{\vec{v}+ \vec{w}} \leq \norm{\vec{v}} + \norm{\vec{w}}\)
            \item \textbf{Cauchy-Schwarz-Ungleichung}:\\
                  \(\abs{\scprod{\vec{v}}{\vec{w}}} \leq \norm{\vec{v}}\cdot\norm{\vec{w}}\)
        \end{enumerate}
    \end{field}
    \begin{field}
        In jedem euklidischen oder unitären Vektorraum \((V, \scprod{\cdot}{\cdot})\) gilt:
        \begin{enumerate}[(i)]
            \item \(\norm{\vec{v}} \geq 0\) für alle \(\vec{v}\in V\) \\
                  \(\norm{\vec{v}} = 0 \Leftrightarrow \vec{v} = \vec{0}\)
            \item \(\norm{s\cdot\vec{v}} = \abs{s}\norm{\vec{v}}\)
            \item \textbf{Dreiecksungleichung}:\\
                  \(\norm{\vec{v}+ \vec{w}} \leq \norm{\vec{v}} + \norm{\vec{w}}\)
            \item \textbf{Cauchy-Schwarz-Ungleichung}:\\
                  \(\abs{\scprod{\vec{v}}{\vec{w}}} \leq \norm{\vec{v}}\cdot\norm{\vec{w}}\)
        \end{enumerate}
        \cite{Satz 10.18}
    \end{field}

    \begin{field}
        In jedem euklidischen oder unitären Vektorraum \((V, \scprod{\cdot}{\cdot})\) gilt:
        \begin{enumerate}[(i)]
            \item \(\norm{\vec{v}} \geq 0\) für alle \(\vec{v}\in V\) \\
                  \(\norm{\vec{v}} = 0 \Leftrightarrow \vec{v} = \vec{0}\)
            \item \(\norm{s\cdot\vec{v}} = \hide{\abs{s}\norm{\vec{v}}}\)
            \item \textbf{Dreiecksungleichung}:\\
                  \(\norm{\vec{v}+ \vec{w}} \leq \norm{\vec{v}} + \norm{\vec{w}}\)
            \item \textbf{Cauchy-Schwarz-Ungleichung}:\\
                  \(\abs{\scprod{\vec{v}}{\vec{w}}} \leq \norm{\vec{v}}\cdot\norm{\vec{w}}\)
        \end{enumerate}
    \end{field}
    \begin{field}
        In jedem euklidischen oder unitären Vektorraum \((V, \scprod{\cdot}{\cdot})\) gilt:
        \begin{enumerate}[(i)]
            \item \(\norm{\vec{v}} \geq 0\) für alle \(\vec{v}\in V\) \\
                  \(\norm{\vec{v}} = 0 \Leftrightarrow \vec{v} = \vec{0}\)
            \item \(\norm{s\cdot\vec{v}} = \abs{s}\norm{\vec{v}}\)
            \item \textbf{Dreiecksungleichung}:\\
                  \(\norm{\vec{v}+ \vec{w}} \leq \norm{\vec{v}} + \norm{\vec{w}}\)
            \item \textbf{Cauchy-Schwarz-Ungleichung}:\\
                  \(\abs{\scprod{\vec{v}}{\vec{w}}} \leq \norm{\vec{v}}\cdot\norm{\vec{w}}\)
        \end{enumerate}
        \cite{Satz 10.18}
    \end{field}

    \begin{field}
        In jedem euklidischen oder unitären Vektorraum \((V, \scprod{\cdot}{\cdot})\) gilt:
        \begin{enumerate}[(i)]
            \item \(\norm{\vec{v}} \geq 0\) für alle \(\vec{v}\in V\) \\
                  \(\norm{\vec{v}} = 0 \Leftrightarrow \vec{v} = \vec{0}\)
            \item \(\norm{s\cdot\vec{v}} = \abs{s}\norm{\vec{v}}\)
            \item \hint{\textbf{Dreiecksungleichung}:}\\
                  \hide{\(\norm{\vec{v}+ \vec{w}} \leq \norm{\vec{v}} + \norm{\vec{w}}\)}
            \item \textbf{Cauchy-Schwarz-Ungleichung}:\\
                  \(\abs{\scprod{\vec{v}}{\vec{w}}} \leq \norm{\vec{v}}\cdot\norm{\vec{w}}\)
        \end{enumerate}
    \end{field}
    \begin{field}
        In jedem euklidischen oder unitären Vektorraum \((V, \scprod{\cdot}{\cdot})\) gilt:
        \begin{enumerate}[(i)]
            \item \(\norm{\vec{v}} \geq 0\) für alle \(\vec{v}\in V\) \\
                  \(\norm{\vec{v}} = 0 \Leftrightarrow \vec{v} = \vec{0}\)
            \item \(\norm{s\cdot\vec{v}} = \abs{s}\norm{\vec{v}}\)
            \item \textbf{Dreiecksungleichung}:\\
                  \(\norm{\vec{v}+ \vec{w}} \leq \norm{\vec{v}} + \norm{\vec{w}}\)
            \item \textbf{Cauchy-Schwarz-Ungleichung}:\\
                  \(\abs{\scprod{\vec{v}}{\vec{w}}} \leq \norm{\vec{v}}\cdot\norm{\vec{w}}\)
        \end{enumerate}
        \cite{Satz 10.18}
    \end{field}
    \begin{field}
        In jedem euklidischen oder unitären Vektorraum \((V, \scprod{\cdot}{\cdot})\) gilt:
        \begin{enumerate}[(i)]
            \item \(\norm{\vec{v}} \geq 0\) für alle \(\vec{v}\in V\) \\
                  \(\norm{\vec{v}} = 0 \Leftrightarrow \vec{v} = \vec{0}\)
            \item \(\norm{s\cdot\vec{v}} = \abs{s}\norm{\vec{v}}\)
            \item \textbf{Dreiecksungleichung}:\\
                  \(\norm{\vec{v}+ \vec{w}} \leq \norm{\vec{v}} + \norm{\vec{w}}\)
            \item \hint{\textbf{Cauchy-Schwarz-Ungleichung}}:\\
                  \hide{\(\abs{\scprod{\vec{v}}{\vec{w}}} \leq \norm{\vec{v}}\cdot\norm{\vec{w}}\)}
        \end{enumerate}
    \end{field}
    \begin{field}
        In jedem euklidischen oder unitären Vektorraum \((V, \scprod{\cdot}{\cdot})\) gilt:
        \begin{enumerate}[(i)]
            \item \(\norm{\vec{v}} \geq 0\) für alle \(\vec{v}\in V\) \\
                  \(\norm{\vec{v}} = 0 \Leftrightarrow \vec{v} = \vec{0}\)
            \item \(\norm{s\cdot\vec{v}} = \abs{s}\norm{\vec{v}}\)
            \item \textbf{Dreiecksungleichung}:\\
                  \(\norm{\vec{v}+ \vec{w}} \leq \norm{\vec{v}} + \norm{\vec{w}}\)
            \item \textbf{Cauchy-Schwarz-Ungleichung}:\\
                  \(\abs{\scprod{\vec{v}}{\vec{w}}} \leq \norm{\vec{v}}\cdot\norm{\vec{w}}\)
        \end{enumerate}
        \cite{Satz 10.18}
    \end{field}
\end{note}


%%%%%%%%% Vorlesung 3 %%%%%%%%%%%%
\begin{note}
    \tags{Def}
    \xplain{1794ed94-bdbc-11ec-9d64-0242ac120002}

    \begin{field}
        Sei \((V,\scprod{\cdot}{\cdot})\) ein euklidischer oder unitärer Vektorraum. \(\vec{v}\in V\) heißt \textbf{normiert}, falls \dots
    \end{field}
    \begin{field}
        Sei \((V,\scprod{\cdot}{\cdot})\) ein euklidischer oder unitärer Vektorraum. \(\vec{v}\in V\) heißt \textbf{normiert}, falls \(\norm{\vec{v}} = 1\)
        \cite{Def. 10.20}
    \end{field}

    \begin{field}
        Sei \((V,\scprod{\cdot}{\cdot})\) ein euklidischer oder unitärer Vektorraum. \(\vec{v},\vec{w}\in V\) sind zueinander \textbf{orthogonal}, falls \dots
    \end{field}
    \begin{field}
        Sei \((V,\scprod{\cdot}{\cdot})\) ein euklidischer oder unitärer Vektorraum. \(\vec{v},\vec{w}\in V\) sind zueinander \textbf{orthogonal}, falls \(\scprod{\vec{v}}{\vec{w}} = 0\)

        [Notation: \(\vec{v}\perp\vec{w}\)]
        \cite{Def. 10.20}
    \end{field}

    \begin{field}
        Sei \((V,\scprod{\cdot}{\cdot})\) ein euklidischer oder unitärer Vektorraum. Eine Basis \(B\) von \(V\) heißt \textbf{Orthonormalbasis} von \(V\), falls \dots
    \end{field}
    \begin{field}
        Sei \((V,\scprod{\cdot}{\cdot})\) ein euklidischer oder unitärer Vektorraum. Eine Basis \(B=(\vec{b}_i)_i\) von \(V\) heißt \textbf{Orthonormalbasis} von \(V\), falls
        \begin{itemize}
            \item jedes \(\vec{b}_i\in B\) normiert ist, und
            \item jeweils \(\vec{b}_i\perp\vec{b}_j\) für \(i\neq j\)
        \end{itemize}
        \cite{Def. 10.20}
    \end{field}
\end{note}

\begin{note}
    \tags{Def}
    \xplain{44bab364-bdbf-11ec-9d64-0242ac120002}

    \begin{field}
        Sei \((V,\scprod{\cdot}{\cdot})\) euklidisch oder unitär. Das \textbf{orthogonale Komplement} eines Untervektorraums \(W\subseteq V\) ist
        \[W^{\perp} :=\phantom{\{\vec{v}\in V \mid \vec{v}\perp\vec{w} \text{ für alle } \vec{w}\in W\}}\]
    \end{field}
    \begin{field}
        Sei \((V,\scprod{\cdot}{\cdot})\) euklidisch oder unitär. Das \textbf{orthogonale Komplement} eines Untervektorraums \(W\subseteq V\) ist
        \[W^{\perp} := \{\vec{v}\in V \mid \vec{v}\perp\vec{w} \text{ für alle } \vec{w}\in W\}\]
        \cite{Def. 10.23}
    \end{field}
\end{note}

\begin{note}
    \tags{Def}
    \xplain{bd8678f8-bdc1-11ec-9d64-0242ac120002}

    \xfield{
        Ein \textbf{affiner Unterraum} eines Vektorraums \(V\) ist \dots
    }
    \begin{field}
        Ein \textbf{affiner Unterraum} eines Vektorraums \(V\) ist eine Teilmenge der Form
        \[\vec{u}_0 + U = \{\vec{v}\in V \mid \vec{v} - \vec{u}_0 \in U\}\]
        für einen Untervektorraum \(U\subseteq V\).
    \end{field}

    \xfield{
        Eine \textbf{affine Hyperebene} ist \dots
    }
    \begin{field}
        Eine \textbf{affine Hyperebene} ist ein affiner Unterraum, dessen zugehöriger Untervektorraum \(U\) die Dimension \(\dim U = \dim V - 1\) hat.
    \end{field}
\end{note}

\begin{note}
    \tags{Satz}
    \xplain{3f20cf54-bdd0-11ec-9d64-0242ac120002}

    \begin{field}
        \textbf{Hessesche Normalform}

        Jede affine Hyperrebene in einem euklidischen oder unitären VR hat die Form
        \[H = \phantom{\{\vec{v}\in V\mid\scprod{\vec{v}}{\vec{n}} = d\}} \]
    \end{field}
    \begin{field}
        \textbf{Hessesche Normalform}

        Jede affine Hyperrebene in einem euklidischen oder unitären VR hat die Form
        \[H = \{\vec{v}\in V\mid\scprod{\vec{v}}{\vec{n}} = d\}\]
        für einen normierten Vektor \(\vec{n}\) und ein \(d\in\RR\) mit \(d\geq 0\)
        \cite{Satz 10.25}
    \end{field}
\end{note}

\begin{note}
    \tags{Satz}
    \xplain{47c00c10-bdd0-11ec-9d64-0242ac120002}

    \begin{field}
        Für \(\vec{x, y}\in\RR^3\) gilt:
        \begin{enumerate}
            \item \hint{Winkel zwischen Vektoren und ihrem Kreuzprodukt} \hide{\((\vec{x}\times\vec{y})\perp\vec{x}\) und
                  \((\vec{x}\times\vec{y})\perp\vec{y}\)}
            \item \(\norm{\vec{x}\times\vec{y}} =
                    \norm{\vec{x}}\cdot\norm{\vec{y}}
                    \cdot \sin\sphericalangle(\vec{x,y})\)
        \end{enumerate}
    \end{field}
    \begin{field}
        Für \(\vec{x, y}\in\RR^3\) gilt:
        \begin{enumerate}
            \item \((\vec{x}\times\vec{y})\perp\vec{x}\) und
                  \((\vec{x}\times\vec{y})\perp\vec{y}\)
            \item \(\norm{\vec{x}\times\vec{y}} =
                    \norm{\vec{x}}\cdot\norm{\vec{y}}
                    \cdot \sin\sphericalangle(\vec{x,y})\)
        \end{enumerate}
        \cite{Satz 10.28}
    \end{field}
    \begin{field}
        Für \(\vec{x, y}\in\RR^3\) gilt:
        \begin{enumerate}
            \item \((\vec{x}\times\vec{y})\perp\vec{x}\) und
                  \((\vec{x}\times\vec{y})\perp\vec{y}\)
            \item \hint{Norm des Kreuzprodukts} \hide{\(\norm{\vec{x}\times\vec{y}} =
                    \norm{\vec{x}}\cdot\norm{\vec{y}}
                    \cdot \sin\sphericalangle(\vec{x,y})\)}
        \end{enumerate}
    \end{field}
    \begin{field}
        Für \(\vec{x, y}\in\RR^3\) gilt:
        \begin{enumerate}
            \item \((\vec{x}\times\vec{y})\perp\vec{x}\) und
                  \((\vec{x}\times\vec{y})\perp\vec{y}\)
            \item \(\norm{\vec{x}\times\vec{y}} =
                    \norm{\vec{x}}\cdot\norm{\vec{y}}
                    \cdot \sin\sphericalangle(\vec{x,y})\)
        \end{enumerate}
        \cite{Satz 10.28}
    \end{field}
\end{note}

\tags{LinA-II-11-Isometrien}
%%%%%%%%% Vorlesung 4 %%%%%%%%%%%%

\begin{note}
    \tags{Def}
    \xplain{282be322-bf03-11ec-9d64-0242ac120002}

    \begin{field}
        Sei \((V,\scprod{\cdot}{\cdot})\) euklidischer oder unitärer Vektorraum.\\
        Eine \textbf{Isometrie} ist eine lineare Abbildung \(f\colon V \rightarrow V\), für die gilt \dots
    \end{field}
    \begin{field}
        Sei \((V,\scprod{\cdot}{\cdot})\) euklidischer oder unitärer Vektorraum.\\
        Eine \textbf{Isometrie} ist eine lineare Abbildung \(f\colon V \rightarrow V\), für die gilt:
        \[\scprod{f(\vec{v})}{f(\vec{w})} = \scprod{\vec{v}}{\vec{w}} \text{ für alle } \vec{v}\in V\]
        \cite{Def. 11.1}
    \end{field}

    \begin{field}
        Sei \((V,\scprod{\cdot}{\cdot})\) euklidischer oder unitärer Vektorraum.\\
        Eine \textbf{Isometrie} ist \hide{eine lineare Abbildung \(f\colon V \rightarrow V\)}, für die gilt:
        \[\scprod{f(\vec{v})}{f(\vec{w})} = \scprod{\vec{v}}{\vec{w}} \text{ für alle } \vec{v}\in V\]
    \end{field}
    \begin{field}
        Sei \((V,\scprod{\cdot}{\cdot})\) euklidischer oder unitärer Vektorraum.\\
        Eine \textbf{Isometrie} ist eine lineare Abbildung \(f\colon V \rightarrow V\), für die gilt:
        \[\scprod{f(\vec{v})}{f(\vec{w})} = \scprod{\vec{v}}{\vec{w}} \text{ für alle } \vec{v}\in V\]
        \cite{Def. 11.1}
    \end{field}
\end{note}

\begin{note}
    \tags{Satz}
    \xplain{21857eac-bf03-11ec-9d64-0242ac120002}

    \xfield{Alle Eigenwerte einer Isometrie \dots}
    \xfield{
        Alle Eigenwerte einer Isometrie haben Betrag 1.
        \cite{Satz 10.2}
    }

    \begin{field}
        Eigenwerte einer Isometrie in einem euklidischen Vektorraum haben die Form
        \[a = \phantom{\pm 1}\]
    \end{field}
    \begin{field}
        Eigenwerte einer Isometrie in einem euklidischen Vektorraum haben die Form
        \[a = \pm 1\]
        \cite{Satz 10.2}
    \end{field}

    \begin{field}
        Eigenwerte einer Isometrie in einem unitären Vektorraum haben die Form
        \[a = \phantom{x + \mathrm{i}y \text{ mit } x^2+y^2=1}\]
    \end{field}
    \begin{field}
        Eigenwerte einer Isometrie in einem unitären Vektorraum haben die Form
        \[a = x + \mathrm{i}y \text{ mit } x^2+y^2=1\]
        \cite{Satz 10.2}
    \end{field}

    \xfield{Eigenvektoren zu verschiedenen Eigenwerten einer Isometrie \dots}
    \xfield{
        Eigenvektoren zu verschiedenen Eigenwerten einer Isometrie stehen senkrecht zueinander.
        \cite{Satz 10.2}
    }
\end{note}

\begin{note}
    \tags{Satz}
    \xplain{59a91c62-bf03-11ec-9d64-0242ac120002}

    \begin{field} %1
        Sei \(K\in\{\RR,\CC\}\) und \(V=K^n\) versehen mit dem entsprechenden Standardskalarprodukt.\\
        Für \(A\in \Mat_K(n\times n)\) sind äquivalent:
        \begin{enumerate}[(i)]
            \item \hide{\(f_A\) ist eine Isometrie auf \(V\).}
            \item \(A\) ist invertierbar und \(A^{-1}=\overline{A}^T\)
            \item Die Spalten von \(A\) bilden eine Orthonormalbasis von \(V\).
            \item Die Zeilen von \(A\) bilden eine Orthonormalbasis von \(V\).
        \end{enumerate}
    \end{field}
    \begin{field}
        Sei \(K\in\{\RR,\CC\}\) und \(V=K^n\) versehen mit dem entsprechenden Standardskalarprodukt.\\
        Für \(A\in \Mat_K(n\times n)\) sind äquivalent:
        \begin{enumerate}[(i)]
            \item \(f_A\) ist eine Isometrie auf \(V\).
            \item \(A\) ist invertierbar und \(A^{-1}=\overline{A}^T\)
            \item Die Spalten von \(A\) bilden eine Orthonormalbasis von \(V\).
            \item Die Zeilen von \(A\) bilden eine Orthonormalbasis von \(V\).
        \end{enumerate}
        \cite{Satz 11.3}
    \end{field}

    \begin{field}%2
        Sei \(K\in\{\RR,\CC\}\) und \(V=K^n\) versehen mit dem entsprechenden Standardskalarprodukt.\\
        Für \(A\in \Mat_K(n\times n)\) sind äquivalent:
        \begin{enumerate}[(i)]
            \item \(f_A\) ist eine Isometrie auf \(V\).
            \item \hide{\(A\) ist invertierbar und \(A^{-1}=\overline{A}^T\)}
            \item Die Spalten von \(A\) bilden eine Orthonormalbasis von \(V\).
            \item Die Zeilen von \(A\) bilden eine Orthonormalbasis von \(V\).
        \end{enumerate}
    \end{field}
    \begin{field}
        Sei \(K\in\{\RR,\CC\}\) und \(V=K^n\) versehen mit dem entsprechenden Standardskalarprodukt.\\
        Für \(A\in \Mat_K(n\times n)\) sind äquivalent:
        \begin{enumerate}[(i)]
            \item \(f_A\) ist eine Isometrie auf \(V\).
            \item \(A\) ist invertierbar und \(A^{-1}=\overline{A}^T\)
            \item Die Spalten von \(A\) bilden eine Orthonormalbasis von \(V\).
            \item Die Zeilen von \(A\) bilden eine Orthonormalbasis von \(V\).
        \end{enumerate}
        \cite{Satz 11.3}
    \end{field}

    \begin{field}%3
        Sei \(K\in\{\RR,\CC\}\) und \(V=K^n\) versehen mit dem entsprechenden Standardskalarprodukt.\\
        Für \(A\in \Mat_K(n\times n)\) sind äquivalent:
        \begin{enumerate}[(i)]
            \item \(f_A\) ist eine Isometrie auf \(V\).
            \item \(A\) ist invertierbar und \(A^{-1}=\overline{A}^T\)
            \item \hide{Die Spalten von \(A\) bilden eine Orthonormalbasis von \(V\).}
            \item Die Zeilen von \(A\) bilden eine Orthonormalbasis von \(V\).
        \end{enumerate}
    \end{field}
    \begin{field}
        Sei \(K\in\{\RR,\CC\}\) und \(V=K^n\) versehen mit dem entsprechenden Standardskalarprodukt.\\
        Für \(A\in \Mat_K(n\times n)\) sind äquivalent:
        \begin{enumerate}[(i)]
            \item \(f_A\) ist eine Isometrie auf \(V\).
            \item \(A\) ist invertierbar und \(A^{-1}=\overline{A}^T\)
            \item Die Spalten von \(A\) bilden eine Orthonormalbasis von \(V\).
            \item Die Zeilen von \(A\) bilden eine Orthonormalbasis von \(V\).
        \end{enumerate}
        \cite{Satz 11.3}
    \end{field}

    \begin{field}%4
        Sei \(K\in\{\RR,\CC\}\) und \(V=K^n\) versehen mit dem entsprechenden Standardskalarprodukt.\\
        Für \(A\in \Mat_K(n\times n)\) sind äquivalent:
        \begin{enumerate}[(i)]
            \item \(f_A\) ist eine Isometrie auf \(V\).
            \item \(A\) ist invertierbar und \(A^{-1}=\overline{A}^T\)
            \item Die Spalten von \(A\) bilden eine Orthonormalbasis von \(V\).
            \item \hide{Die Zeilen von \(A\) bilden eine Orthonormalbasis von \(V\).}
        \end{enumerate}
    \end{field}
    \begin{field}
        Sei \(K\in\{\RR,\CC\}\) und \(V=K^n\) versehen mit dem entsprechenden Standardskalarprodukt.\\
        Für \(A\in \Mat_K(n\times n)\) sind äquivalent:
        \begin{enumerate}[(i)]
            \item \(f_A\) ist eine Isometrie auf \(V\).
            \item \(A\) ist invertierbar und \(A^{-1}=\overline{A}^T\)
            \item Die Spalten von \(A\) bilden eine Orthonormalbasis von \(V\).
            \item Die Zeilen von \(A\) bilden eine Orthonormalbasis von \(V\).
        \end{enumerate}
        \cite{Satz 11.3}
    \end{field}
\end{note}

%%%%%%%%% Vorlesung 5 %%%%%%%%%%%%
\begin{note}
    \tags{Def Satz}
    \xplain{91122bd6-c30d-11ec-9d64-0242ac120002}
    \begin{field} %1
        Die \textbf{allgemeine lineare Gruppe} über einem Körper \(K\) ist definiert als
        \begin{align*}
            \text{GL}_n(K) :&= \hide{(\{A\in\Mat_K(n \times n) \mid f_A \text{ Isomorphismus} \}, \cdot)}\\
            &= (\{A\in\Mat_K(n \times n) \mid \det(A) \neq 0 \}, \cdot)
        \end{align*}
    \end{field}
    \begin{field}
        Die \textbf{allgemeine lineare Gruppe} über einem Körper \(K\) ist definiert als
        \begin{align*}
            \text{GL}_n(K) :&= (\{A\in\Mat_K(n \times n) \mid f_A \text{ Isomorphismus} \}, \cdot)\\
            &= (\{A\in\Mat_K(n \times n) \mid \det(A) \neq 0 \}, \cdot)
        \end{align*}
        \cite{Def/Satz 11.5}
    \end{field}

    \begin{field} %2
        Die \textbf{allgemeine lineare Gruppe} über einem Körper \(K\) ist definiert als
        \begin{align*}
            \text{GL}_n(K) :&= (\{A\in\Mat_K(n \times n) \mid  f_A \text{ Isomorphismus} \}, \cdot)\\
            &= \hide{(\{A\in\Mat_K(n \times n) \mid \det(A) \neq 0 \}, \cdot)}
        \end{align*}
    \end{field}
    \begin{field}
        Die \textbf{allgemeine lineare Gruppe} über einem Körper \(K\) ist definiert als
        \begin{align*}
            \text{GL}_n(K) :&= (\{A\in\Mat_K(n \times n) \mid  f_A \text{ Isomorphismus} \}, \cdot)\\
            &= (\{A\in\Mat_K(n \times n) \mid  \det(A) \neq 0 \}, \cdot)
        \end{align*}
        \cite{Def/Satz 11.5}
    \end{field}

    \begin{field} %3
        Die \textbf{spezielle lineare Gruppe} über einem Körper \(K\) ist definiert als
        \[\text{SL}_n(K) := \phantom{(\{ A \in\Mat_K(n\times n)\mid \det(A) = 1\}, \cdot ) }\]
    \end{field}
    \begin{field}
        Die \textbf{spezielle lineare Gruppe} über einem Körper \(K\) ist definiert als
        \[\text{SL}_n(K) := (\{ A \in\Mat_K(n\times n)\mid \det(A) = 1\}, \cdot )\]
        \cite{Def/Satz 11.5}
    \end{field}
\end{note}

\begin{note}
    \tags{Def Satz}
    \xplain{9a5ba794-c30d-11ec-9d64-0242ac120002}
    \begin{field}
        Die \textbf{orthogonale Gruppe} ist definiert als
        \begin{align*}
            \text{O}(n) :&= \hide{(\{A\in\Mat_{\RR}(n\times n) \mid f_A \text{ Isometrie}\}, \cdot)}\\
            &= (\{A \in \text{GL}_n(\RR) \mid A^{-1} = A^T\}, \cdot)
        \end{align*}
    \end{field}
    \begin{field}
        Die \textbf{orthogonale Gruppe} ist definiert als
        \begin{align*}
            \text{O}(n) :&= (\{A\in\Mat_{\RR}(n\times n) \mid f_A \text{ Isometrie}\}, \cdot)\\
            &= (\{A \in \text{GL}_n(\RR) \mid A^{-1} = A^T\}, \cdot)
        \end{align*}
        \cite{Def/Satz 11.5}
    \end{field}

    \begin{field}
        Die \textbf{orthogonale Gruppe} ist definiert als
        \begin{align*}
            \text{O}(n) :&= (\{A\in\Mat_{\RR}(n\times n) \mid f_A \text{ Isometrie}\}, \cdot)\\
            &= \hide{(\{A \in \text{GL}_n(\RR) \mid A^{-1} = A^T\}, \cdot)}
        \end{align*}
    \end{field}
    \begin{field}
        Die \textbf{orthogonale Gruppe} ist definiert als
        \begin{align*}
            \text{O}(n) :&= (\{A\in\Mat_{\RR}(n\times n) \mid f_A \text{ Isometrie}\}, \cdot)\\
            &= (\{A \in \text{GL}_n(\RR) \mid A^{-1} = A^T\}, \cdot)
        \end{align*}
        \cite{Def/Satz 11.5}
    \end{field}

    \begin{field}
        Die \textbf{spezielle orthogonale Gruppe} ist definiert als
        \[\text{SO}(n) := \phantom{\text{O}(n) \cap \text{SL}_n(\RR)}\]
    \end{field}
    \begin{field}
        Die \textbf{spezielle orthogonale Gruppe} ist definiert als
        \[\text{SO}(n) := \text{O}(n) \cap \text{SL}_n(\RR)\]
        \cite{Def/Satz 11.5}
    \end{field}
\end{note}

\begin{note}
    \tags{Def Satz}
    \xplain{a0b73478-c30d-11ec-9d64-0242ac120002}
    \begin{field}
        Die \textbf{unitäre Gruppe} ist definiert als
        \begin{align*}
            \text{U}(n) :&= \hide{(\{A\in\Mat_{\CC}(n\times n) \mid f_A \text{ Isometrie}\}, \cdot)}\\
            &= (\{A \in \text{GL}_n(\CC) \mid A^{-1} = \overline{A}^T\}, \cdot)
        \end{align*}
    \end{field}
    \begin{field}
        Die \textbf{unitäre Gruppe} ist definiert als
        \begin{align*}
            \text{U}(n) :&= (\{A\in\Mat_{\CC}(n\times n) \mid f_A \text{ Isometrie}\}, \cdot)\\
            &= (\{A \in \text{GL}_n(\CC) \mid A^{-1} = \overline{A}^T\}, \cdot)
        \end{align*}
        \cite{Def/Satz 11.5}
    \end{field}

    \begin{field}
        Die \textbf{unitäre Gruppe} ist definiert als
        \begin{align*}
            \text{U}(n) :&= (\{A\in\Mat_{\CC}(n\times n) \mid f_A \text{ Isometrie}\}, \cdot)\\
            &= \hide{(\{A \in \text{GL}_n(\CC) \mid A^{-1} = \overline{A}^T\}, \cdot)}
        \end{align*}
    \end{field}
    \begin{field}
        Die \textbf{unitäre Gruppe} ist definiert als
        \begin{align*}
            \text{U}(n) :&= (\{A\in\Mat_{\CC}(n\times n) \mid f_A \text{ Isometrie}\}, \cdot)\\
            &= (\{A \in \text{GL}_n(\CC) \mid A^{-1} = \overline{A}^T\}, \cdot)
        \end{align*}
        \cite{Def/Satz 11.5}
    \end{field}

    \begin{field}
        Die \textbf{spezielle unitäre Gruppe} ist definiert als
        \[\text{SU}(n) := \phantom{\text{U}(n) \cap \text{SL}_n(\CC)}\]
    \end{field}
    \begin{field}
        Die \textbf{spezielle unitäre Gruppe} ist definiert als
        \[\text{SU}(n) := \text{U}(n) \cap \text{SL}_n(\RR)\]
        \cite{Def/Satz 11.5}
    \end{field}
\end{note}

\begin{note}
    \xplain{ac148780-c30d-11ec-9d64-0242ac120002}
    \begin{field}
        Eine Isometrie auf \(\RR^2\) ist \hide{eine Rotation um \(\vec{0}\)} oder eine Spiegelung an einer Ursprungsgeraden.
    \end{field}
    \begin{field}
        Eine Isometrie auf \(\RR^2\) ist eine Rotation um \(\vec{0}\) oder eine Spiegelung an einer Ursprungsgeraden.
        \cite{Lemma 11.6}
    \end{field}

    \begin{field}
        Eine Isometrie auf \(\RR^2\) ist eine Rotation um \(\vec{0}\) oder \hide{eine Spiegelung an einer Ursprungsgeraden}.
    \end{field}
    \begin{field}
        Eine Isometrie auf \(\RR^2\) ist eine Rotation um \(\vec{0}\) oder eine Spiegelung an einer Ursprungsgeraden.
        \cite{Lemma 11.6}
    \end{field}

    \begin{field}
        Die orthogonale Gruppe \(\text{O}(2)\) hat die Form
        \begin{align*}
            \text{O}(2) = &\dots\\
            \phantom{\dot\cup} &\left\{\left(\begin{smallmatrix}
                              a & b\\ b & -a
                          \end{smallmatrix}\right) \;\middle\vert\; a,b\in \RR, a^2+b^2=1\right\}
        \end{align*}
    \end{field}
    \begin{field}
        Die orthogonale Gruppe \(\text{O}(2)\) hat die Form
        \begin{align*}
            \text{O}(2) = &\left\{\left(\begin{smallmatrix}
                                  a & -b\\ b & a
                                \end{smallmatrix}\right) \;\middle\vert\; a,b\in \RR, a^2+b^2=1\right\}\\
            \dot\cup &\left\{\left(\begin{smallmatrix}
                              a & b\\ b & -a
                            \end{smallmatrix}\right) \;\middle\vert\; a,b\in \RR, a^2+b^2=1\right\}
        \end{align*}
        \cite{Lemma 11.6}
    \end{field}

    \begin{field}
        Die orthogonale Gruppe \(\text{O}(2)\) hat die Form
        \begin{align*}
            \text{O}(2) = &\left\{\left(\begin{smallmatrix}
                                  a & -b\\ b & a
                                \end{smallmatrix}\right) \;\middle\vert\; a,b\in \RR, a^2+b^2=1\right\}\\
            \phantom{\dot\cup} &\dots
        \end{align*}
    \end{field}
    \begin{field}
        Die orthogonale Gruppe \(\text{O}(2)\) hat die Form
        \begin{align*}
            \text{O}(2) = &\left\{\left(\begin{smallmatrix}
                                  a & -b\\ b & a
                                \end{smallmatrix}\right) \;\middle\vert\; a,b\in \RR, a^2+b^2=1\right\}\\
            \dot\cup &\left\{\left(\begin{smallmatrix}
                              a & b\\ b & -a
                          \end{smallmatrix}\right) \;\middle\vert\; a,b\in \RR, a^2+b^2=1\right\}
        \end{align*}
        \cite{Lemma 11.6}
    \end{field}
\end{note}

\begin{note}
    \renewcommand{\cite}[1]{\bigskip\hfill{\color{gray}\tiny\(\to\) #1}}
    \tags{Satz}
    \xplain{b2f42114-c30d-11ec-9d64-0242ac120002}
    \begin{field}
        \textbf{Struktursatz für euklidische Isometrien}

        Jede Isometrie eines \hide{endlich-dimensionalen} euklidischen Vektorraums hat bezüglich einer geeigneten \emph{Orthonormal}basis eine darstellende Matrix der Form
        \[\left(\begin{smallmatrix}
            +1 \\
              & \ddots \\
              &        & +1 &  &       &  &     \mbox{\Large 0}\\
              &        &    &-1\\
              &        &    &  &\ddots\\
              &        &    &  &       &-1\\
              &\mbox{\Large 0}&    &  &       &  & A_1\\
              &        &    &  &       &  &     &\ddots\\
              &        &    &  &       &  &     &       & A_k\\
        \end{smallmatrix}\right)\]
        mit \(A_i\) Rotationsmatrizen.
    \end{field}
    \begin{field}
        \textbf{Struktursatz für euklidische Isometrien}

        Jede Isometrie eines endlich-dimensionalen euklidischen Vektorraums hat bezüglich einer geeigneten \emph{Orthonormal}basis eine darstellende Matrix der Form
        \[\left(\begin{smallmatrix}
            +1 \\
              & \ddots \\
              &        & +1 &  &       &  &     \mbox{\Large 0}\\
              &        &    &-1\\
              &        &    &  &\ddots\\
              &        &    &  &       &-1\\
              &\mbox{\Large 0}&    &  &       &  & A_1\\
              &        &    &  &       &  &     &\ddots\\
              &        &    &  &       &  &     &       & A_k\\
        \end{smallmatrix}\right)\]
        mit \(A_i\) Rotationsmatrizen.
        \cite{satz 11.7}
    \end{field}

    \begin{field}
        \textbf{Struktursatz für euklidische Isometrien}

        Jede Isometrie eines endlich-dimensionalen euklidischen Vektorraums hat \hide{bezüglich einer geeigneten \emph{Orthonormal}basis} eine darstellende Matrix der Form
        \[\left(\begin{smallmatrix}
            +1 \\
              & \ddots \\
              &        & +1 &  &       &  &     \mbox{\Large 0}\\
              &        &    &-1\\
              &        &    &  &\ddots\\
              &        &    &  &       &-1\\
              &\mbox{\Large 0}&    &  &       &  & A_1\\
              &        &    &  &       &  &     &\ddots\\
              &        &    &  &       &  &     &       & A_k\\
        \end{smallmatrix}\right)\]
        mit \(A_i\) Rotationsmatrizen.
    \end{field}
    \begin{field}
        \textbf{Struktursatz für euklidische Isometrien}

        Jede Isometrie eines endlich-dimensionalen euklidischen Vektorraums hat bezüglich einer geeigneten \emph{Orthonormal}basis eine darstellende Matrix der Form
        \[\left(\begin{smallmatrix}
            +1 \\
              & \ddots \\
              &        & +1 &  &       &  &     \mbox{\Large 0}\\
              &        &    &-1\\
              &        &    &  &\ddots\\
              &        &    &  &       &-1\\
              &\mbox{\Large 0}&    &  &       &  & A_1\\
              &        &    &  &       &  &     &\ddots\\
              &        &    &  &       &  &     &       & A_k\\
        \end{smallmatrix}\right)\]
        mit \(A_i\) Rotationsmatrizen.
        \cite{satz 11.7}
    \end{field}

    \begin{field}
        \textbf{Struktursatz für euklidische Isometrien}

        Jede Isometrie eines endlich-dimensionalen euklidischen Vektorraums hat bezüglich einer geeigneten \emph{Orthonormal}basis eine darstellende Matrix der Form \dots
    \end{field}
    \begin{field}
        \textbf{Struktursatz für euklidische Isometrien}

        Jede Isometrie eines endlich-dimensionalen euklidischen Vektorraums hat bezüglich einer geeigneten \emph{Orthonormal}basis eine darstellende Matrix der Form
        \[\left(\begin{smallmatrix}
            +1 \\
              & \ddots \\
              &        & +1 &  &       &  &     \mbox{\large 0}\\
              &        &    &-1\\
              &        &    &  &\ddots\\
              &        &    &  &       &-1\\
              &\mbox{\large 0}&    &  &       &  & A_1\\
              &        &    &  &       &  &     &\ddots\\
              &        &    &  &       &  &     &       & A_k\\
        \end{smallmatrix}\right)\]
        mit \(A_i\) Rotationsmatrizen. \cite{satz 11.7}
    \end{field}
\end{note}

\begin{note}
    \tags{Def}
    \xplain{b9b834fe-c30d-11ec-9d64-0242ac120002}

    \begin{field}
        Sei \(K\) eine Körper, \(V\) ein K-Vektorraum, \(f\colon V\rightarrow V\) ein Endomorphismus.\\
        Ein Untervektorraum \(W \subseteq V\) heißt \textbf{f-stabil}, falls \dots
    \end{field}
    \begin{field}
        Sei \(K\) eine Körper, \(V\) ein K-Vektorraum, \(f\colon V\rightarrow V\) ein Endomorphismus.\\
        Ein Untervektorraum \(W \subseteq V\) heißt \textbf{f-stabil}, falls $f(W)\subseteq W$.
        \cite{Satz 11.7}
    \end{field}
\end{note}


%%%%%%%%% Vorlesung 6 %%%%%%%%%%%%

\begin{note}
    \tags{Lemma}
    \xplain{e16fae78-c78b-11ec-9d64-0242ac120002}

    \begin{field}
        Jede Isometrie $f$ eines endlich-dimensionalen euklidischen Vektorraums $V\neq \{ 0\}$ besitzt \dots \hint{Untervektorraum}
    \end{field}
    \begin{field}
        Jede Isometrie $f$ eines endlich-dimensionalen euklidischen Vektorraums $V\neq \{ 0\}$ besitzt einen $f$-stabilen Untervektorraum der Dimension 1 oder 2.
        \cite{Lemma 11.11}
    \end{field}
\end{note}
\begin{note}
    \tags{Satz}
    \xplain{950750fa-c78a-11ec-9d64-0242ac120002}
    \begin{field}
        \textbf{Struktursatz für unitäre Isometrien}

        Jede Isometrie eines \hide{endlich-dimensionalen} unitären Vektorraums wird bezüglich einer geeigneten \emph{Orthonormal}basis dargestellt von einer Diagonalmatrix
        \[\begin{pmatrix}
             a_1 \\
            &\ddots\\
            &&a_n
        \end{pmatrix}\]
        mit $a_i \in \CC, \vert a_i \vert = 1$.
    \end{field}
    \begin{field}
        \textbf{Struktursatz für unitäre Isometrien}

        Jede Isometrie eines endlich-dimensionalen unitären Vektorraums wird bezüglich einer geeigneten \emph{Orthonormal}basis dargestellt von einer Diagonalmatrix
        \[\begin{pmatrix}
             a_1 \\
            &\ddots\\
            &&a_n
        \end{pmatrix}\]
        mit $a_i \in \CC, \vert a_i \vert = 1$.
        \cite{Satz 11.12}
    \end{field}

    \begin{field}
        \textbf{Struktursatz für unitäre Isometrien}

        Jede Isometrie eines endlich-dimensionalen unitären Vektorraums wird \hide{bezüglich einer geeigneten \emph{Orthonormal}basis} dargestellt von einer Diagonalmatrix
        \[\begin{pmatrix}
             a_1 \\
            &\ddots\\
            &&a_n
        \end{pmatrix}\]
        mit $a_i \in \CC, \vert a_i \vert = 1$.
    \end{field}
    \begin{field}
        \textbf{Struktursatz für unitäre Isometrien}

        Jede Isometrie eines endlich-dimensionalen unitären Vektorraums wird bezüglich einer geeigneten \emph{Orthonormal}basis dargestellt von einer Diagonalmatrix
        \[\begin{pmatrix}
             a_1 \\
            &\ddots\\
            &&a_n
        \end{pmatrix}\]
        mit $a_i \in \CC, \vert a_i \vert = 1$.
        \cite{Satz 11.12}
    \end{field}

    \begin{field}
        \textbf{Struktursatz für unitäre Isometrien}

        Jede Isometrie eines endlich-dimensionalen unitären Vektorraums wird bezüglich einer geeigneten \emph{Orthonormal}basis dargestellt von \dots
    \end{field}
    \begin{field}
        \textbf{Struktursatz für unitäre Isometrien}

        Jede Isometrie eines endlich-dimensionalen unitären Vektorraums wird bezüglich einer geeigneten \emph{Orthonormal}basis dargestellt von einer Diagonalmatrix
        \[\begin{pmatrix}
             a_1 \\
            &\ddots\\
            &&a_n
        \end{pmatrix}\]
        mit $a_i \in \CC, \vert a_i \vert = 1$.
        \cite{Satz 11.12}
    \end{field}
\end{note}

\tags{LinA-II-12-Hauptachsentransformation}

\begin{note}
    \tags{Def}
    \xplain{f145cd78-c790-11ec-9d64-0242ac120002}

    \begin{field}
        Ein Endomorphismus $f$ eines euklidischen oder unitären Vektorraums $(V, \scprod{\cdot}{\cdot})$ ist \emph{selbstadjungiert}, falls \dots
    \end{field}
    \begin{field}
        Ein Endomorphismus $f$ eines euklidischen oder unitären Vektorraums $(V, \scprod{\cdot}{\cdot})$ ist \emph{selbstadjungiert}, falls
        \[\scprod{f(\vec{v})}{\vec{w}} = \scprod{\vec{v}}{f(\vec{w})} \quad \forall \vec{v,w}\in V\]
        \cite{Def. 12.1}
    \end{field}

    \begin{field}
        Sei $A\in \Mat_K(n\times n), K\in\{\RR,\CC\}$. Dann sind äquivalent:
        \begin{itemize}
            \item \hide{Die lineare Abbildung $f_A: K^n\longrightarrow K^n$ ist \emph{selbstadjungiert} bezüglich des Standardskalarprodukts auf $K^n$.}
            \item $A$ ist symmetrisch ($A = A^T$) \\ bzw. hermitesch ($A = \overline{A^T}$)
        \end{itemize}
    \end{field}
    \begin{field}
        Sei $A\in \Mat_K(n\times n), K\in\{\RR,\CC\}$. Dann sind äquivalent:
        \begin{itemize}
            \item Die lineare Abbildung $f_A: K^n\longrightarrow K^n$ ist \emph{selbstadjungiert} bezüglich des Standardskalarprodukts auf $K^n$.
            \item $A$ ist symmetrisch ($A = A^T$) \\ bzw. hermitesch ($A = \overline{A^T}$)
        \end{itemize}
        \cite{Notiz 12.2}
    \end{field}

    \begin{field}
        Sei $A\in \Mat_K(n\times n), K\in\{\RR,\CC\}$. Dann sind äquivalent:
        \begin{itemize}
            \item Die lineare Abbildung $f_A: K^n\longrightarrow K^n$ ist \emph{selbstadjungiert} bezüglich des Standardskalarprodukts auf $K^n$.
            \item \hide{$A$ ist symmetrisch ($A = A^T$) \\ bzw. hermitesch ($A = \overline{A^T}$}
        \end{itemize}
    \end{field}
    \begin{field}
        Sei $A\in \Mat_K(n\times n), K\in\{\RR,\CC\}$. Dann sind äquivalent:
        \begin{itemize}
            \item Die lineare Abbildung $f_A: K^n\longrightarrow K^n$ ist \emph{selbstadjungiert} bezüglich des Standardskalarprodukts auf $K^n$.
            \item $A$ ist symmetrisch ($A = A^T$) \\ bzw. hermitesch ($A = \overline{A^T}$)
        \end{itemize}
        \cite{Notiz 12.2}
    \end{field}
\end{note}

\begin{note}
    \tags{Satz}
    \xplain{eaa8db5e-c790-11ec-9d64-0242ac120002}
    \begin{field}
        Alle Eigenwerte eines \emph{selbstadjungierten} Endomorphismus \dots
    \end{field}
    \begin{field}
        Alle Eigenwerte eines \emph{selbstadjungierten} Endomorphismus sind reell.
        \cite{Satz 12.3}
    \end{field}

    \begin{field}
        Eigenvektoren zu verschiedenen Eigenwerten eines \emph{selbstadjungierten} Endomorphismus \dots
    \end{field}
    \begin{field}
        Eigenvektoren zu verschiedenen Eigenwerten eines \emph{selbstadjungierten} Endomorphismus stehen senkrecht zueinander.
        \cite{Satz 12.3}
    \end{field}
\end{note}

%%%%%%%%% Vorlesung 7 %%%%%%%%%%%%
\begin{note}
    \tags{Satz}
    \xplain{c8642aef-feb0-4273-9657-9874c2229640}
    \begin{field}
        \textbf{Spektralsatz\\}

        Sei $(V, \scprod{\cdot}{\cdot})$ ein endlich-dimensionaler euklidischer oder unitärer Vektorraum.

        Zu \hide{jedem selbstadjungierten Endomorphismus} $f$ auf $V$ existiert eine Orthonormalbasis von $V$ aus Eigenvektoren von $f$.
    \end{field}
    \begin{field}
        \textbf{Spektralsatz\\ (Hauptachsentransformation für selbstadj. Abb.)}

        Sei $(V, \scprod{\cdot}{\cdot})$ ein endlich-dimensionaler euklidischer oder unitärer Vektorraum.

        Zu jedem selbstadjungierten Endomorphismus $f$ auf $V$ existiert eine Orthonormalbasis von $V$ aus Eigenvektoren von $f$.
        \cite{Satz 12.4}
    \end{field}

    \begin{field}
        \textbf{Spektralsatz\\ (Hauptachsentransformation für selbstadj. Abb.)}

        Sei $(V, \scprod{\cdot}{\cdot})$ ein endlich-dimensionaler euklidischer oder unitärer Vektorraum.

        Zu jedem selbstadjungierten Endomorphismus $f$ auf $V$ existiert \hide{eine Orthonormalbasis von $V$ aus Eigenvektoren von $f$.}
    \end{field}
    \begin{field}
        \textbf{Spektralsatz\\ (Hauptachsentransformation für selbstadj. Abb.)}

        Sei $(V, \scprod{\cdot}{\cdot})$ ein endlich-dimensionaler euklidischer oder unitärer Vektorraum.

        Zu jedem selbstadjungierten Endomorphismus $f$ auf $V$ existiert eine Orthonormalbasis von $V$ aus Eigenvektoren von $f$.
        \cite{Satz 12.4}
    \end{field}
\end{note}

\begin{note}
    \tags{Satz}
    \xplain{c7c37423-b07c-4459-8002-a0327be7c516}

    \begin{field}
        \textbf{Hauptachsentransformation für Matrizen}

        Zu jeder \hide{\textbf{reellen symmetrischen}} Matrix $A$ existiert eine Matrix $S\in\text{O}(n)$ mit
        \[S^{-1}AS = S^TAS = \begin{pmatrix}
            a_1&&\text{\huge 0}\\
            & \ddots&\\
            \smash{\text{\huge 0}}&& a_n
        \end{pmatrix}\]
        für gewisse $a_i\in\RR$.
    \end{field}
    \begin{field}
        \textbf{Hauptachsentransformation für Matrizen}

        Zu jeder \textbf{reellen symmetrischen} Matrix\\
         $A\in\Mat_{\RR}(n\times n)$ existiert eine Matrix $S\in\text{O}(n)$ mit
        \[S^{-1}AS = S^TAS = \begin{pmatrix}
            a_1&&\text{\huge 0}\\
            & \ddots&\\
            \smash{\text{\huge 0}}&& a_n
        \end{pmatrix}\]
        für gewisse $a_i\in\RR$.
        \cite{Satz 12.5}
    \end{field}

    \begin{field}
        \textbf{Hauptachsentransformation für Matrizen}

        Zu jeder \textbf{reellen symmetrischen} Matrix\\
        $A\in\Mat_{\RR}(n\times n)$ existiert eine Matrix \hide{$S\in\text{O}(n)$} mit
        \[S^{-1}AS = S^TAS = \begin{pmatrix}
            a_1&&\text{\huge 0}\\
            & \ddots&\\
            \smash{\text{\huge 0}}&& a_n
        \end{pmatrix}\]
        für gewisse $a_i\in\RR$.
    \end{field}
    \begin{field}
        \textbf{Hauptachsentransformation für Matrizen}

        Zu jeder \textbf{reellen symmetrischen} Matrix\\
        $A\in\Mat_{\RR}(n\times n)$ existiert eine Matrix $S\in\text{O}(n)$ mit
        \[S^{-1}AS = S^TAS = \begin{pmatrix}
            a_1&&\text{\huge 0}\\
            & \ddots&\\
            \smash{\text{\huge 0}}&& a_n
        \end{pmatrix}\]
        für gewisse $a_i\in\RR$.
        \cite{Satz 12.5}
    \end{field}

    \begin{field}
        \textbf{Hauptachsentransformation für Matrizen}

        Zu jeder \textbf{reellen symmetrischen} Matrix\\
        $A\in\Mat_{\RR}(n\times n)$ existiert eine Matrix $S\in\text{O}(n)$ mit \dots
    \end{field}
    \begin{field}
        \textbf{Hauptachsentransformation für Matrizen}

        Zu jeder \textbf{reellen symmetrischen} Matrix\\
        $A\in\Mat_{\RR}(n\times n)$ existiert eine Matrix $S\in\text{O}(n)$ mit
        \[S^{-1}AS = S^TAS = \begin{pmatrix}
            a_1&&\text{\huge 0}\\
            & \ddots&\\
            \smash{\text{\huge 0}}&& a_n
        \end{pmatrix}\]
        für gewisse $a_i\in\RR$.
        \cite{Satz 12.5}
    \end{field}
\end{note}

\begin{note}
    \tags{Satz}
    \xplain{6b695356-9bb9-415a-b938-df47fa36982d}

    \begin{field}
        \textbf{Hauptachsentransformation für Matrizen}

        Zu jeder \hide{\textbf{hermiteschen}} Matrix $A\in\Mat_{\CC}(n\times n)$ existiert eine Matrix $S\in\text{U}(n)$ mit
        \[S^{-1}AS = \bar{S}^TAS = \begin{pmatrix}
            a_1&&\text{\huge 0}\\
            & \ddots&\\
            \smash{\text{\huge 0}}&& a_n
        \end{pmatrix}\]
        für gewisse $a_i\in\RR$.
    \end{field}
    \begin{field}
        \textbf{Hauptachsentransformation für Matrizen}

        Zu jeder \textbf{hermiteschen} Matrix $A\in\Mat_{\CC}(n\times n)$ existiert eine Matrix $S\in\text{U}(n)$ mit
        \[S^{-1}AS = \overline{S}^TAS = \begin{pmatrix}
            a_1&&\text{\huge 0}\\
            & \ddots&\\
            \smash{\text{\huge 0}}&& a_n
        \end{pmatrix}\]
        für gewisse $a_i\in\RR$.
        \cite{Satz 12.5}
    \end{field}

    \begin{field}
        \textbf{Hauptachsentransformation für Matrizen}

        Zu jeder \textbf{hermiteschen} Matrix $A\in\Mat_{\CC}(n\times n)$ existiert eine Matrix \hide{$S\in\text{U}(n)$} mit
        \[S^{-1}AS = \bar{S}^TAS = \begin{pmatrix}
            a_1&&\text{\huge 0}\\
            & \ddots&\\
            \smash{\text{\huge 0}}&& a_n
        \end{pmatrix}\]
        für gewisse $a_i\in\RR$.
    \end{field}
    \begin{field}
        \textbf{Hauptachsentransformation für Matrizen}

        Zu jeder \textbf{hermiteschen} Matrix $A\in\Mat_{\CC}(n\times n)$ existiert eine Matrix $S\in\text{U}(n)$ mit
        \[S^{-1}AS = \overline{S}^TAS = \begin{pmatrix}
            a_1&&\text{\huge 0}\\
            & \ddots&\\
            \smash{\text{\huge 0}}&& a_n
        \end{pmatrix}\]
        für gewisse $a_i\in\RR$.
        \cite{Satz 12.5}
    \end{field}

    \begin{field}
        \textbf{Hauptachsentransformation für Matrizen}

        Zu jeder \textbf{hermiteschen} Matrix $A\in\Mat_{\CC}(n\times n)$ existiert eine Matrix $S\in\text{U}(n)$ mit \dots
    \end{field}
    \begin{field}
        \textbf{Hauptachsentransformation für Matrizen}

        Zu jeder \textbf{hermiteschen} Matrix $A\in\Mat_{\CC}(n\times n)$ existiert eine Matrix $S\in\text{U}(n)$ mit
        \[S^{-1}AS = \overline{S}^TAS = \begin{pmatrix}
            a_1&&\text{\huge 0}\\
            & \ddots&\\
            \smash{\text{\huge 0}}&& a_n
        \end{pmatrix}\]
        für gewisse $a_i\in\RR$.
        \cite{Satz 12.5}
    \end{field}
\end{note}

\begin{note}
    \tags{Satz}
    \xplain{812a96d5-4323-43af-80d4-1227d16c208f}

    \begin{field}
        \textbf{Hauptachsentransformation für Formen}

        Zu jeder symmetrischen Bilinearform $\beta$ auf einem endlich-dimensionalen euklidischen Vektorraum $(V,\scprod{\cdot}{\cdot})$ existiert \hide{eine Orthonormalbasis} $B$ mit
        \[M_B(\beta) = \begin{pmatrix}
            a_1&&\text{\huge 0}\\
            & \ddots&\\
            \smash{\text{\huge 0}}&& a_n
        \end{pmatrix}\]
        für gewisse $a_i\in\RR$.
    \end{field}
    \begin{field}
        \textbf{Hauptachsentransformation für Formen}

        Zu jeder symmetrischen Bilinearform $\beta$ auf einem endlich-dimensionalen euklidischen Vektorraum $(V,\scprod{\cdot}{\cdot})$ existiert eine Orthonormalbasis $B$, in der gilt:
        \[M_B(\beta) = \begin{pmatrix}
            a_1&&\text{\huge 0}\\
            & \ddots&\\
            \smash{\text{\huge 0}}&& a_n
        \end{pmatrix}\]
        für gewisse $a_i\in\RR$.
        \cite{Satz 12.6}
    \end{field}

    \begin{field}
        \textbf{Hauptachsentransformation für Formen}

        Zu jeder symmetrischen Bilinearform $\beta$ auf einem endlich-dimensionalen euklidischen Vektorraum $(V,\scprod{\cdot}{\cdot})$ existiert eine Orthonormalbasis $B$, in der gilt: \dots
    \end{field}
    \begin{field}
        \textbf{Hauptachsentransformation für Formen}

        Zu jeder symmetrischen Bilinearform $\beta$ auf einem endlich-dimensionalen euklidischen Vektorraum $(V,\scprod{\cdot}{\cdot})$ existiert eine Orthonormalbasis $B$, in der gilt:
        \[M_B(\beta) = \begin{pmatrix}
            a_1&&\text{\huge 0}\\
            & \ddots&\\
            \smash{\text{\huge 0}}&& a_n
        \end{pmatrix}\]
        für gewisse $a_i\in\RR$.
        \cite{Satz 12.6}
    \end{field}

    \begin{field}
        \textbf{Hauptachsentransformation für Formen}

        Zu jeder hermiteschen Sesquilinearform $\beta$ auf einem endlich-dimensionalen unitären Vektorraum $(V,\scprod{\cdot}{\cdot})$ existiert \hide{eine Orthonormalbasis $B$} mit:
        \[M_B(\beta) = \begin{pmatrix}
            a_1&&\text{\huge 0}\\
            & \ddots&\\
            \smash{\text{\huge 0}}&& a_n
        \end{pmatrix}\]
        für gewisse $a_i\in\RR$.
    \end{field}
    \begin{field}
        \textbf{Hauptachsentransformation für Formen}

        Zu jeder hermiteschen Sesquilinearform $\beta$ auf einem endlich-dimensionalen unitären Vektorraum $(V,\scprod{\cdot}{\cdot})$ existiert eine Orthonormalbasis $B$, in der gilt:
        \[M_B(\beta) = \begin{pmatrix}
            a_1&&\text{\huge 0}\\
            & \ddots&\\
            \smash{\text{\huge 0}}&& a_n
        \end{pmatrix}\]
        für gewisse $a_i\in\RR$.
        \cite{Satz 12.6}
    \end{field}

    \begin{field}
        \textbf{Hauptachsentransformation für Formen}

        Zu jeder hermiteschen Sesquilinearform $\beta$ auf einem endlich-dimensionalen unitären Vektorraum $(V,\scprod{\cdot}{\cdot})$ existiert eine Orthonormalbasis $B$, in der gilt: \dots
    \end{field}
    \begin{field}
        \textbf{Hauptachsentransformation für Formen}

        Zu jeder hermiteschen Sesquilinearform $\beta$ auf einem endlich-dimensionalen unitären Vektorraum $(V,\scprod{\cdot}{\cdot})$ existiert eine Orthonormalbasis $B$, in der gilt:
        \[M_B(\beta) = \begin{pmatrix}
            a_1&&\text{\huge 0}\\
            & \ddots&\\
            \smash{\text{\huge 0}}&& a_n
        \end{pmatrix}\]
        für gewisse $a_i\in\RR$.
        \cite{Satz 12.6}
    \end{field}
\end{note}

\begin{note}
    \tags{Def}
    \xplain{304525a1-aaed-4d28-93f4-7be8c6841bb1}

    \begin{field}
        Sei $V$ ein endlich-dimensionaler reeller Vektorraum und $\beta$ eine Bilinearform auf $V$.\\
        Die assoziierte \textbf{quadratische Abbildung} ist \dots
    \end{field}
    \begin{field}
        Sei $V$ ein endlich-dimensionaler reeller Vektorraum und $\beta$ eine Bilinearform auf $V$.\\
        Die assoziierte \textbf{quadratische Abbildung} ist
        \begin{align*}
            q_{\beta}: &V \longrightarrow \RR\\
            &\vec{v} \mapsto \beta(\vec{v},\vec{v})
        \end{align*}
        \cite{Def 12.7}
    \end{field}

    \begin{field}
        Sei $V$ ein endlich-dimensionaler reeller Vektorraum und $\beta$ eine Bilinearform auf $V$.\\
        Die assoziierte \textbf{reelle affine Quadrik} ist \dots
    \end{field}
    \begin{field}
        Sei $V$ ein endlich-dimensionaler reeller Vektorraum und $\beta$ eine Bilinearform auf $V$.\\
        Die assoziierte \textbf{reelle affine Quadrik} ist die Menge
        \[Q_{\beta} := \{\vec{v}\in V \ \vert \  \beta(\vec{v},\vec{v}) = 1\}\]
        \cite{Def 12.7}
    \end{field}
\end{note}

\begin{note}
    \tags{Satz}
    \xplain{3754bb2e-25f8-4d9a-bc3e-5418a61ed780}

    \begin{field}
        \textbf{Hauptachsentransformation für Quadriken}

        Jede reelle affine Quadrik in einem euklidischen Vektorraum $(V,\scprod{\cdot}{\cdot})$ hat bezüglich einer geeigneten Orthonormalbasis $(\vec{b_1,\dots,b_n})$ von $V$ die Form \dots
    \end{field}
    \begin{field}
        \textbf{Hauptachsentransformation für Quadriken}

        Jede reelle affine Quadrik in einem \textbf{euklidischen} Vektorraum $(V,\scprod{\cdot}{\cdot})$ hat bezüglich einer geeigneten Orthonormalbasis $(\vec{b_1,\dots,b_n})$ von $V$ die Form
        \[Q = \{\sum_{i=1}^n x_i\vec{b_i}\in V \ \vert \ \sum a_ix_i^2 = 1\}\]
        für gewisse $a_i\in \RR$.
        \cite{Satz 12.8}
    \end{field}
\end{note}

\begin{note}
    \tags{Satz}
    \xplain{b5290b76-3aa7-435f-bbb1-6b0a9d3294ee}

    \begin{field}
        \textbf{Trägheitssatz von Sylvester}

        Jede reelle symmetrische Matrix $A$ ist kongruent zu einer Diagonalmatrix der Form \dots
    \end{field}
    \begin{field}
        \textbf{Trägheitssatz von Sylvester}

        \small{Jede reelle symmetrische Matrix $A$ ist kongruent zu einer Diagonalmatrix der Form}
        \[\left(\begin{smallmatrix}
            +1\\
            &\ddots\\
            &&+1\\
            &&&-1\\
            &&&&\ddots\\
            &&&&&-1\\
            &&&&&&0\\
            &&&&&&&\ddots\\
            &&&&&&&&0
        \end{smallmatrix}\right)\]
        \small{Die Anzahl der $+1$-, $-1$- und $0$-Einträge ist dabei durch $A$ eindeutig bestimmt.}
        \cite{Satz 12.9}
    \end{field}
\end{note}

%%%%%%%%% Vorlesung 8 %%%%%%%%%%%%
\tags{LinA-II-13-Euklidische-Ringe}

\begin{note}
    \tags{Def}
    \xplain{55ce91b8-e23b-4abe-9697-514111245c46}

    \begin{field}
        Ein \textbf{Integritätsring} ist ein \hide{kommutativer} Ring $R$, in dem für alle $a,b\in R$ gilt:
        \[ab = 0 \Rightarrow [a= 0 \text{ oder } b=0] \]
    \end{field}
    \begin{field}
        Ein \textbf{Integritätsring} ist ein kommutativer Ring $R$, in dem für alle $a,b\in R$ gilt:
        \[ab = 0 \Rightarrow [a= 0 \text{ oder } b=0] \]
        \cite{Def. 13.1}
    \end{field}
    \begin{field}
        Ein \textbf{Integritätsring} ist ein kommutativer Ring $R$, in dem gilt: \dots

    \end{field}
    \begin{field}
        Ein \textbf{Integritätsring} ist ein kommutativer Ring $R$, in dem für alle $a,b\in R$ gilt:
        \[ab = 0 \Rightarrow [a= 0 \text{ oder } b=0] \]
        \cite{Def. 13.1}
    \end{field}
\end{note}

\begin{note}
    \tags{Satz}
    \xplain{14fc167f-f4e0-4a6d-bfd8-cab8ec0f0377}

    \begin{field}
        Für jeden Integritätsring $R$ ist \hint{Polynomring über $R$}\hide{auch $R[X]$ ein Integritätsring}.
    \end{field}
    \begin{field}
        Für jeden Integritätsring $R$ ist auch $R[X]$ ein Integritätsring.
        \cite{Satz 13.2}
    \end{field}

    \begin{field}
        Für jeden Integritätsring $R$ gilt
        \[(R[X])^\times = \phantom{R^{\times}}\]
    \end{field}
    \begin{field}
        Für jeden Integritätsring $R$ gilt
        \[(R[X])^\times = R^{\times}\]
        \cite{Satz 13.2}
    \end{field}
\end{note}


\begin{note}
    \tags{Def}
    \xplain{cbe89edb-ace5-43ba-afc9-5fa41ab1b1cb}

    \begin{field}
        Sei $R$ ein Integritätsring und $a,b\in R$.\\
        $a$ ist ein \textbf{Teiler} von $b$ und $b$ ist ein \textbf{Vielfaches} von $a$ ($a\vert b$) genau dann, wenn \dots
    \end{field}
    \begin{field}
        Sei $R$ ein Integritätsring und $a,b\in R$.\\
        $a$ ist ein \textbf{Teiler} von $b$ und $b$ ist ein \textbf{Vielfaches} von $a$ ($a\vert b$) genau dann, wenn
        \[\exists c\in R \colon b = c\cdot a\]
        \cite{Def. 13.4}
    \end{field}

    \begin{field}
        Sei $R$ ein Integritätsring und $a,b\in R$.\\
        $a$ und $b$ sind \textbf{assoziiert} ($a \sim b$) genau dann, wenn \dots
    \end{field}
    \begin{field}
        Sei $R$ ein Integritätsring und $a,b\in R$.\\
        $a$ und $b$ sind \textbf{assoziiert} ($a \sim b$) genau dann, wenn
        \[\exists c\in R^\times\colon b = c\cdot a\]
        \cite{Def. 13.4}
    \end{field}
\end{note}

\begin{note}
    \tags{Def}
    \xplain{3024e3e1-e61b-40fe-890f-52ea6401dd61}

    \begin{field}
        Sei $R$ ein Integritätsring und $a,b\in R$.\\
        $c$ ist ein \textbf{größter gemeinsamer Teiler} von $a$ und $b$ ($c \sim \text{ggT}(a,b)$) genau dann, wenn
        \begin{itemize}
            \item \hide{$c\vert a$ und $c\vert b$}
        \end{itemize}
        und
        \begin{itemize}
            \item $\forall c'\in R\colon (c'\vert a) \text{ und } (c'\vert b) \Rightarrow c'\vert c$
        \end{itemize}
    \end{field}
    \begin{field}
        Sei $R$ ein Integritätsring und $a,b\in R$.\\
        $c$ ist ein \textbf{größter gemeinsamer Teiler} von $a$ und $b$ ($c \sim \text{ggT}(a,b)$) genau dann, wenn
        \begin{itemize}
            \item $c\vert a$ und $c\vert b$
        \end{itemize}
        und
        \begin{itemize}
            \item $\forall c'\in R\colon (c'\vert a) \text{ und } (c'\vert b) \Rightarrow c'\vert c$
        \end{itemize}
        \cite{Def. 13.7}
    \end{field}

    \begin{field}
        Sei $R$ ein Integritätsring und $a,b\in R$.\\
        $c$ ist ein \textbf{größter gemeinsamer Teiler} von $a$ und $b$ ($c \sim \text{ggT}(a,b)$) genau dann, wenn
        \begin{itemize}
            \item $c\vert a$ und $c\vert b$
        \end{itemize}
        und
        \begin{itemize}
            \item \hide{$\forall c'\in R\colon (c'\vert a) \text{ und } (c'\vert b) \Rightarrow c'\vert c$}
        \end{itemize}
    \end{field}
    \begin{field}
        Sei $R$ ein Integritätsring und $a,b\in R$.\\
        $c$ ist ein \textbf{größter gemeinsamer Teiler} von $a$ und $b$ ($c \sim \text{ggT}(a,b)$) genau dann, wenn
        \begin{itemize}
            \item $c\vert a$ und $c\vert b$
        \end{itemize}
        und
        \begin{itemize}
            \item $\forall c'\in R\colon (c'\vert a) \text{ und } (c'\vert b) \Rightarrow c'\vert c$
        \end{itemize}
        \cite{Def. 13.7}
    \end{field}

    \begin{field}
        Sei $R$ ein Integritätsring und $a,b\in R$.\\
        $a$ und $b$ sind \textbf{teilerfremd}, falls \dots
    \end{field}
    \begin{field}
        Sei $R$ ein Integritätsring und $a,b\in R$.\\
        $a$ und $b$ sind \textbf{teilerfremd}, falls $1\sim \text{ggT}(a,b)$
        \cite{Def. 13.7}
    \end{field}
\end{note}

\begin{note}
    \tags{Def}
    \xplain{c29e019b-0c2c-4dec-94e0-316d6e0f0112}

    \begin{field}
        Sei $R$ ein Integritätsring und $a,b\in R$.\\
        $c$ ist ein \textbf{kleinstes gemeinsames Vielfaches} von $a$ und $b$ ($c \sim \text{kgV}(a,b)$) genau dann, wenn
        \begin{itemize}
            \item \hide{$a\vert c$ und $b\vert c$}
        \end{itemize}
        und
        \begin{itemize}
            \item $\forall c'\in R\colon (a\vert c') \text{ und } (b\vert c') \Rightarrow c\vert c'$
        \end{itemize}
    \end{field}
    \begin{field}
        Sei $R$ ein Integritätsring und $a,b\in R$.\\
        $c$ ist ein \textbf{kleinstes gemeinsames Vielfaches} von $a$ und $b$ ($c \sim \text{kgV}(a,b)$) genau dann, wenn
        \begin{itemize}
            \item $a\vert c$ und $b\vert c$
        \end{itemize}
        und
        \begin{itemize}
            \item $\forall c'\in R\colon (a\vert c') \text{ und } (b\vert c') \Rightarrow c\vert c'$
        \end{itemize}
        \cite{Def. 13.7}
    \end{field}

    \begin{field}
        Sei $R$ ein Integritätsring und $a,b\in R$.\\
        $c$ ist ein \textbf{kleinstes gemeinsames Vielfaches} von $a$ und $b$ ($c \sim \text{kgV}(a,b)$) genau dann, wenn
        \begin{itemize}
            \item $a\vert c$ und $b\vert c$
        \end{itemize}
        und
        \begin{itemize}
            \item \hide{$\forall c'\in R\colon (a\vert c') \text{ und } (b\vert c') \Rightarrow c\vert c'$}
        \end{itemize}
    \end{field}
    \begin{field}
        Sei $R$ ein Integritätsring und $a,b\in R$.\\
        $c$ ist ein \textbf{kleinstes gemeinsames Vielfaches} von $a$ und $b$ ($c \sim \text{kgV}(a,b)$) genau dann, wenn
        \begin{itemize}
            \item $a\vert c$ und $b\vert c$
        \end{itemize}
        und
        \begin{itemize}
            \item $\forall c'\in R\colon (a\vert c') \text{ und } (b\vert c') \Rightarrow c\vert c'$
        \end{itemize}
        \cite{Def. 13.7}
    \end{field}
\end{note}

\begin{note}
    \tags{Def}
    \xplain{f86a676e-ef48-4e9b-a39b-9bf39ad9c931}

    \begin{field}
        Ein Integritätsring $R$ ist \textbf{euklidisch}, falls eine Abbildung
        \[\delta: R\setminus \{0\} \longrightarrow \NN_0\]
        mit folgender Eigenschaft existiert:\\
        Für $a,b\in R$ mit $b\neq 0$ existieren $q,r$ mit        \begin{center}\dots\end{center}

        und
        \[r=0 \text{ oder } \delta(r) < \delta(b)\]
    \end{field}
    \begin{field}
        Ein Integritätsring $R$ ist \textbf{euklidisch}, falls eine Abbildung
        \[\delta: R\setminus \{0\} \longrightarrow \NN_0\]
        mit folgender Eigenschaft existiert:\\
        Für $a,b\in R$ mit $b\neq 0$ existieren $q,r$ mit
        \[ a = q\cdot b + r\]
        und
        \[r=0 \text{ oder } \delta(r) < \delta(b)\]
    \end{field}

    \begin{field}
        Ein Integritätsring $R$ ist \textbf{euklidisch}, falls eine Abbildung
        \[\delta: R\setminus \{0\} \longrightarrow \NN_0\]
        mit folgender Eigenschaft existiert:\\
        Für $a,b\in R$ mit $b\neq 0$ existieren $q,r$ mit
        \[ a = q\cdot b + r\]
        und
        \begin{center}\dots\end{center}
    \end{field}
    \begin{field}
        Ein Integritätsring $R$ ist \textbf{euklidisch}, falls eine Abbildung
        \[\delta: R\setminus \{0\} \longrightarrow \NN_0\]
        mit folgender Eigenschaft existiert:\\
        Für $a,b\in R$ mit $b\neq 0$ existieren $q,r$ mit
        \[ a = q\cdot b + r\]
        und
        \[r=0 \text{ oder } \delta(r) < \delta(b)\]
        \cite{Def. 13.9}
    \end{field}
\end{note}

%%%%%%%%% Vorlesung 9 %%%%%%%%%%%%
\begin{note}
    \tags{Satz}
    \xplain{f84dc2fc-8eb1-4c7b-a9f5-aafbb11fc382}

    \begin{field}
        \textbf{Lemma von Bézout}\\
        In jedem euklidischen Ring gilt:
        \[c \sim \text{ggT}(a,b) \Rightarrow \phantom{\exists x,y\colon c = xa + yb}\]
    \end{field}
    \begin{field}
        \textbf{Lemma von Bézout}\\
        In jedem euklidischen Ring gilt:
        \[c \sim \text{ggT}(a,b) \Rightarrow  \exists x,y \colon c = xa + yb\]
        \cite{Lemma 13.13}
    \end{field}

    \begin{field}
        In jedem euklidischen Ring gilt:
        \[a, b \text{ teilerfremd } \Leftrightarrow \phantom{\exists x,y\colon 1=x\cdot a + y\cdot b}\]
    \end{field}
    \begin{field}
        In jedem euklidischen Ring gilt:
        \[a, b \text{ teilerfremd } \Leftrightarrow \exists x,y\colon 1=x\cdot a + y\cdot b\]
        \cite{Korollar 13.14}
    \end{field}

    \begin{field}
        In jedem euklidischen Ring gilt:
        \[\phantom{a, b \text{ teilerfremd }} \Leftrightarrow \exists x,y\colon 1=x\cdot a + y\cdot b\]
    \end{field}
    \begin{field}
        In jedem euklidischen Ring gilt:
        \[a, b \text{ teilerfremd } \Leftrightarrow \exists x,y\colon 1=x\cdot a + y\cdot b\]
        \cite{Korollar 13.14}
    \end{field}
\end{note}

\begin{note}
    \tags{Def}
    \xplain{1a1b1607-0506-4438-b2a4-e0f89c3a2089}

    \xfield{
        Sei $R$ ein Integritätsring.\\
        Ein Element $p \in R\setminus (R^\times \cup \{0\})$ ist \textbf{irreduzibel}, falls \dots
    }
    \begin{field}
        Sei $R$ ein Integritätsring.\\
        Ein Element $p \in R\setminus (R^\times \cup \{0\})$ ist \textbf{irreduzibel}, falls für $a,b\in R$ gilt:
        \[p = a\cdot b \Rightarrow (a\in R^\times \text{ oder } b\in R^\times)\]
        \cite{Def. 13.15}
    \end{field}

    \xfield{
    Sei $R$ ein Integritätsring.\\
    Ein Element $p \in R\setminus (R^\times \cup \{0\})$ ist \textbf{prim}, falls \dots
    }
    \begin{field}
        Sei $R$ ein Integritätsring.\\
        Ein Element $p \in R\setminus (R^\times \cup \{0\})$ ist \textbf{prim}, falls für $a,b\in R$ gilt:
        \[p\vert ab \Rightarrow p\vert b \text{ oder } p\vert a\]
        \cite{Def. 13.15}
    \end{field}
\end{note}

\begin{note}
    \tags{Satz}
    \xplain{53b1954d-e3f5-4cd5-aa5c-d8266a638202}

    \xfield{
    In einem Integritätsring $R$ gilt: \hint{Zusammenhang prim und irreduzibel}\dots
    }
    \begin{field}
        In einem Integritätsring $R$ gilt:
        \[p \in R \text{ prim } \Rightarrow p \text{ irreduzibel}\]
        \cite{Satz 13.16}
    \end{field}

    \xfield{
    In einem euklidischen Ring $R$ gilt: \hint{Zusammenhang prim und irreduzibel}
    }\dots
    \begin{field}
        In einem euklidischen Ring $R$ gilt:
        \[p \in R \text{ prim } \Leftrightarrow p \text{ irreduzibel}\]
        \cite{Satz 13.16}
    \end{field}
\end{note}

%%%%%%%%% Vorlesung 10 %%%%%%%%%%%%
\begin{note}
    \tags{Def}
    \xplain{e5c8519c-02d7-4561-8a22-6aba87daa2b3}

    \xfield{Eine \textbf{Primfaktorzerlegung} von $a\in R$ ist \dots}
    \begin{field}
        Eine \textbf{Primfaktorzerlegung} von $a\in R$ ist eine Darstellung von $a$ als Produkt
        \[a = p_1p_2\cdots p_r\]
        mit $r\in\NN$ und $p_i\in R$ prim.
        \cite{Def. 13.19}
    \end{field}

    \xfield{Ein Integritätsring $R$ heißt \textbf{faktoriell},\\
    wenn \dots}
    \begin{field}
        Ein Integritätsring $R$ heißt \textbf{faktoriell},\\
        wenn jedes $a\in R\setminus(R^\times \cup \{0\})$ eine Primfaktorzerlegung besitzt.
        \cite{Def. 13.19}
    \end{field}
\end{note}

\begin{note}
    \tags{Satz}
    \xplain{89327ded-c7e4-49f1-a10f-9199de4016be}

    \xfield{Falls eine \emph{Primfaktorzerlegung} von $a\in R$ existiert, so ist diese \dots}
    \begin{field}
        Falls eine \emph{Primfaktorzerlegung} von $a\in R$ existiert, so ist diese \emph{eindeutig} bis auf Reihenfolge der Faktoren und Assoziiertheit.
        \cite{Satz 13.20}
    \end{field}
\end{note}

\begin{note}
    \tags{Satz}
    \xplain{8a2baefd-95e9-4ac4-91d6-96ab9a9a7d0c}

    \xfield{Für jeden Körper $K$ ist \hide{$K[X]$} faktoriell.}
    \begin{field}
        Für jeden Körper $K$ ist $K[X]$ faktoriell.
        \cite{Satz 13.22}
    \end{field}
\end{note}


%%%%%%%%% Vorlesung 11 %%%%%%%%%%%%
\tags{LinA-II-14-Minimalpolynom}

\begin{note}
    \tags{Def}
    \xplain{ec41baef-a06e-43ff-a6d0-24b3ae262be4}

    \begin{field}
        Das \textbf{Minimalpolynom} von $f$ ist das eindeutige Polynom $\mu_f\in K[X]\setminus{\{0\}}$ für das gilt:
        \begin{enumerate}[(1)]
            \item \hide{$\mu_f(f)=0$ (Nullabbildung in $\text{End}_K(V))$}
            \item Unter allen Polynomen $\neq 0$, die (1) erfüllen, hat $\mu_f$ minimalen Grad.
            \item $\mu_f$ ist normiert (d.h. Leitkoeffizient = 1)
        \end{enumerate}
    \end{field}
    \begin{field}
        Das \textbf{Minimalpolynom} von $f$ ist das eindeutige Polynom $\mu_f\in K[X]\setminus{\{0\}}$ für das gilt:
        \begin{enumerate}[(1)]
            \item $\mu_f(f)=0$ (Nullabbildung in $\text{End}_K(V))$
            \item Unter allen Polynomen $\neq 0$, die (1) erfüllen, hat $\mu_f$ minimalen Grad.
            \item $\mu_f$ ist normiert (d.h. Leitkoeffizient = 1)
        \end{enumerate}
        \cite{Def. 14.4}
    \end{field}

    \begin{field}
        Das \textbf{Minimalpolynom} von $f$ ist das eindeutige Polynom $\mu_f\in K[X]\setminus{\{0\}}$ für das gilt:
        \begin{enumerate}[(1)]
            \item $\mu_f(f)=0$ (Nullabbildung in $\text{End}_K(V))$
            \item \hide{Unter allen Polynomen $\neq 0$, die (1) erfüllen, hat $\mu_f$ minimalen Grad.}
            \item $\mu_f$ ist normiert (d.h. Leitkoeffizient = 1)
        \end{enumerate}
    \end{field}
    \begin{field}
        Das \textbf{Minimalpolynom} von $f$ ist das eindeutige Polynom $\mu_f\in K[X]\setminus{\{0\}}$ für das gilt:
        \begin{enumerate}[(1)]
            \item $\mu_f(f)=0$ (Nullabbildung in $\text{End}_K(V))$
            \item Unter allen Polynomen $\neq 0$, die (1) erfüllen, hat $\mu_f$ minimalen Grad.
            \item $\mu_f$ ist normiert (d.h. Leitkoeffizient = 1)
        \end{enumerate}
        \cite{Def. 14.4}
    \end{field}

    \begin{field}
        Das \textbf{Minimalpolynom} von $f$ ist das eindeutige Polynom $\mu_f\in K[X]\setminus{\{0\}}$ für das gilt:
        \begin{enumerate}[(1)]
            \item $\mu_f(f)=0$ (Nullabbildung in $\text{End}_K(V))$
            \item Unter allen Polynomen $\neq 0$, die (1) erfüllen, hat $\mu_f$ minimalen Grad.
            \item \hide{$\mu_f$ ist normiert (d.h. Leitkoeffizient = 1)}
        \end{enumerate}
    \end{field}
    \begin{field}
        Das \textbf{Minimalpolynom} von $f$ ist das eindeutige Polynom $\mu_f\in K[X]\setminus{\{0\}}$ für das gilt:
        \begin{enumerate}[(1)]
            \item $\mu_f(f)=0$ (Nullabbildung in $\text{End}_K(V))$
            \item Unter allen Polynomen $\neq 0$, die (1) erfüllen, hat $\mu_f$ minimalen Grad.
            \item $\mu_f$ ist normiert (d.h. Leitkoeffizient = 1)
        \end{enumerate}
        \cite{Def. 14.4}
    \end{field}
\end{note}

\begin{note}
    \tags{Satz}
    \xplain{b684bb58-b59c-4e97-953f-6d32f2a58dcf}

    \xfield{
    \textbf{Satz von Caley-Hamilton}\\
    Für das charakteristische Polynom eines Endomorphismus $f$ gilt \dots
    }
    \begin{field}
        \textbf{Satz von Caley-Hamilton}\\
        Für das charakteristische Polynom eines Endomorphismus $f$ gilt
        \[\chi_f(f) = 0\]
        \cite{Satz 14.7}
    \end{field}

    \xfield{
    \textbf{Satz von Caley-Hamilton im zyklischen Fall}\\
    Ist $W\subseteq V$ f-zyklisch, so ist\\
     \dots\\
    das Minimalpolynom für $f\vert_W$.
    }
    \begin{field}
        \textbf{Satz von Caley-Hamilton im zyklischen Fall}\\
        Ist $W\subseteq V$ f-zyklisch, so ist
        \[(-1)^{\text{dim} W} \chi_{f\vert_W}\]
        das Minimalpolynom für $f\vert_W$.
        \cite{Satz 14.13}
    \end{field}
\end{note}

\begin{note}
    \tags{Def}
    \xplain{6088f132-4411-44e9-aa19-9c0cfa13ab91}

    \xfield{
    Ein Untervektorraum $W\subseteq V$ ist \textbf{f-zyklisch}, falls \dots
    }
    \begin{field}
         Ein Untervektorraum $W\subseteq V$ ist \textbf{f-zyklisch}, falls
         \[W = \langle \vec{w}, f(\vec{w}), f^2(\vec{w}), \dots \rangle \text{ für ein } \vec{w}\in V\]
         \cite{Def. 14.8}
    \end{field}
\end{note}

\begin{note}
    \tags{Satz}
    \xplain{31141f5d-f74d-45d8-bd71-4577c54a1e2c}

    \xfield{
    Ist $W$ ein \emph{f-zyklischer} Untervektorraum der Dimension $d$, so ist \hint{Basis}\hide{$(\vec{w}, f(\vec{w}), f^2(\vec{w}),\dots, f^{d-1}(\vec{w}))$ eine Basis von $W$.}
    }
    \begin{field}
        Ist $W$ ein \emph{f-zyklischer} Untervektorraum der Dimension $d$, so ist $(\vec{w}, f(\vec{w}), f^2(\vec{w}),\dots, f^{d-1}(\vec{w}))$ eine Basis von $W$.
        \cite{Lemma 14.10}
    \end{field}
\end{note}

\begin{note}
    \tags{Def}
    \xplain{2e432a06-1654-4799-8785-7f9fe0a985e8}

    \xfield{
    Die \textbf{Begleitmatrix} zu einem normierten Polynom\\$A = X^d + \sum_{i=0}^{d-1} a_i X^i$ ist die Matrix \dots
    }
    \begin{field}
        Die \textbf{Begleitmatrix} zu einem normierten Polynom\\$A = X^d + \sum_{i=0}^{d-1} a_i X^i$ ist die Matrix
        \[\begin{pmatrix}
            0 &   &   && -a_0\\
            1 & 0 & &\smash{\text{\huge 0}} & -a_1\\
              & 1 & \ddots\\
              &   & \ddots & 0 & -a_{d-2}\\
              \smash{\text{\huge 0}}&   &   & 1 & -a_{d-1}
    \end{pmatrix}
        \]
    \cite{Def 14.11}
    \end{field}
\end{note}

\begin{note}
    \tags{Satz}
    \xplain{66558bcf-5640-4356-9392-437b27624b3d}

    \xfield{ Sei $f$ ein $V$-Endomorphismus.\\
    Ein Untervektorraum $W\subseteq V$ ist genau dann \hide{f-zyklisch}, wenn er f-stabil ist und eine Basis besitzt, in der $f\vert_W$ durch eine Begleitmatrix gegeben ist.}
    \begin{field}
        Sei $f$ ein $V$-Endomorphismus.\\
       Ein Untervektorraum $W\subseteq V$ ist genau dann f-zyklisch, wenn er f-stabil ist und eine Basis besitzt, in der $f\vert_W$ durch eine Begleitmatrix gegeben ist.
       \cite{Satz 14.12}
    \end{field}

    \xfield{ Sei $f$ ein $V$-Endomorphismus.\\
    Ein Untervektorraum $W\subseteq V$ ist genau dann f-zyklisch, wenn \hide{er f-stabil ist und eine Basis besitzt, in der $f\vert_W$ durch eine Begleitmatrix gegeben ist.}
    }
    \begin{field}
        Sei $f$ ein $V$-Endomorphismus.\\
       Ein Untervektorraum $W\subseteq V$ ist genau dann f-zyklisch, wenn er f-stabil ist und eine Basis besitzt, in der $f\vert_W$ durch eine Begleitmatrix gegeben ist.
       \cite{Satz 14.12}
    \end{field}
\end{note}


%%%%%%%%% Vorlesung 12 %%%%%%%%%%%%
\begin{note}
    \tags{Satz}
    \xplain{707fad0f-485b-4dae-a099-bc9f6c2859b1}

    \xfield{
    \textbf{Spaltungssatz}\\
    Sei $f$ ein Endomorphismus auf einem Vektorraum $V$.\\
    Ist $\mu_f = P\cdot Q$ für zwei teilerfremde normierte Polynome $P$ und $Q$, so ist
    \[V = \phantom{W_P \oplus W_Q}\]
    }
    \begin{field}
        \textbf{Spaltungssatz}\\
        Sei $f$ ein Endomorphismus auf einem Vektorraum $V$.\\
        Ist $\mu_f = P\cdot Q$ für zwei teilerfremde normierte Polynome $P$ und $Q$, so ist \[V = W_P \oplus W_Q\]
        für zwei $f$-stabile Untervektorräume $W_P$ und $W_Q$.
        \cite{Satz 14.17}
    \end{field}

    \begin{field}
        \textbf{Spaltungssatz}\\
        Sei $f$ ein Endomorphismus auf einem Vektorraum $V$.\\
        Ist $\mu_f = P\cdot Q$ für zwei teilerfremde normierte Polynome $P$ und $Q$, so ist \[V = W_P \oplus W_Q\]
        für zwei $f$-stabile Untervektorräume $W_P$ und $W_Q$, für die gilt:
        \[W_P = \phantom{\ker(P(f)) = \im(Q(f))} \text{ und } \mu_f\vert_{W_P} = P\]
        \[W_Q = \phantom{\ker(Q(f)) = \im(P(f))} \text{ und } \mu_f\vert_{W_Q} = Q\]
    \end{field}
    \begin{field}
        \textbf{Spaltungssatz}\\
        Sei $f$ ein Endomorphismus auf einem Vektorraum $V$.\\
        Ist $\mu_f = P\cdot Q$ für zwei teilerfremde normierte Polynome $P$ und $Q$, so ist \[V = W_P \oplus W_Q\]
        für zwei \emph{$f$-stabile} Untervektorräume $W_P$ und $W_Q$, für die gilt:
        \[W_P = \ker(P(f)) = \im(Q(f)) \text{ und } \mu_f\vert_{W_P} = P\]
        \[W_Q = \ker(Q(f)) = \im(P(f)) \text{ und } \mu_f\vert_{W_Q} = Q\]
        \cite{Satz 14.17}
    \end{field}

    \begin{field}
        \textbf{Spaltungssatz}\\
        Sei $f$ ein Endomorphismus auf einem Vektorraum $V$.\\
        Ist $\mu_f = P\cdot Q$ für zwei teilerfremde normierte Polynome $P$ und $Q$, so ist \[V = W_P \oplus W_Q\]
        für zwei \emph{$f$-stabile} Untervektorräume $W_P$ und $W_Q$, für die gilt:
        \[W_P = \ker(P(f)) = \im(Q(f)) \text{ und } \dots\]
        \[W_Q = \ker(Q(f)) = \im(P(f)) \text{ und } \dots\]
    \end{field}
    \begin{field}
        \textbf{Spaltungssatz}\\
        Sei $f$ ein Endomorphismus auf einem Vektorraum $V$.\\
        Ist $\mu_f = P\cdot Q$ für zwei teilerfremde normierte Polynome $P$ und $Q$, so ist \[V = W_P \oplus W_Q\]
        für zwei \emph{$f$-stabile} Untervektorräume $W_P$ und $W_Q$, für die gilt:
        \[W_P = \ker(P(f)) = \im(Q(f)) \text{ und } \mu_f\vert_{W_P} = P\]
        \[W_Q = \ker(Q(f)) = \im(P(f)) \text{ und } \mu_f\vert_{W_Q} = Q\]
        \cite{Satz 14.17}
    \end{field}
\end{note}

%%%%%%%%% Vorlesung 13 %%%%%%%%%%%%

\begin{note}
    \tags{Satz}
    \xplain{18d801db-b3d0-4036-bfb0-0472478225c4}

    \xfield{
    \textbf{Drittes Diagonalisierbarkeitskriterium}\\
    Ein Endomorphismus $f$ ist diagonalisierbar genau dann, wenn \hint{Minimalpolynom}\dots
    }

    \begin{field}

        \textbf{Drittes Diagonalisierbarkeitskriterium}\\
        Ein Endomorphismus $f$ ist diagonalisierbar genau dann, wenn $\mu_f$ in paarweise verschiedene Linearfaktoren zerfällt.
        \cite{Korollar 14.19}
    \end{field}
\end{note}

\tags{LinA-II-15-Jordannormalform}
\begin{note}
    \tags{Def}
    \xplain{2c897805-8ca5-4437-8781-b1ad98648537}

    \xfield{ Ein \textbf{Jordanblock} ist \dots}
    \begin{field}
        Ein \textbf{Jordanblock} ist eine (Unter-)matrix der Form
        \[J(m; a) := \begin{pmatrix}
            a& 1\\
            &a&1&&\smash{\text{\huge 0}}\\
            &&\ddots&\ddots\\
            &&&\ddots&1\\
            \smash{\text{\huge 0}}&&&&a
        \end{pmatrix} \in \Mat_K(m\times m)\]
        \cite{Def 15.1}
    \end{field}

    \xfield{Ein \textbf{Hauptraumblock} ist \dots}
    \begin{field}
        \setlength{\fboxsep}{0pt}
        Ein \textbf{Hauptraumblock} ist eine (Unter-)matrix der Form
        \small
        \[ H(m_1,\dots,m_k;a) := \left(\begin{smallmatrix}
            \fbox{\, $J(m_1;a)$}\\
            &\fbox{\, $J(m_2;a)$}&&\smash{\text{\huge 0}}\\
            && \ddots\\
            \smash{\text{\huge 0}}&&&\fbox{\, $J(m_k;a)$}
        \end{smallmatrix}\right)\]
        \cite{Def 15.1}
    \end{field}
\end{note}

\begin{note}
    \tags{Satz}
    \xplain{15f3d72f-70b5-495b-bb04-a3f157dbea7b}
    \xfield{
    \textbf{Jordannormalform}\\
    Sei $f$ ein Endomorphismus eines endlich-dimensionalen Vektorraums.\\
    Zerfällt $\chi_f$ in Linearfaktoren, so hat $f$ bezüglich einer geeigneten Basis folgende Gestalt:\\
    \dots
    }
    \begin{field}
        \setlength{\fboxsep}{0pt}
        \textbf{Jordannormalform}\\
        Sei $f$ ein Endomorphismus eines endlich-dimensionalen Vektorraums.\\
        Zerfällt $\chi_f$ in Linearfaktoren, so hat $f$ bezüglich einer geeigneten Basis folgende Gestalt:
        \[_B M_B(f) = \left(\begin{smallmatrix}
        \fbox{\, $H(\dots;a_1)$}\\
        &\fbox{\, $H(\dots;a_2)$}&&\smash{\text{\huge 0}}\\
        &&\ddots\\
        \smash{\text{\huge 0}}&&&\fbox{\, $H(\dots;a_l)$}
        \end{smallmatrix}\right)\]
        Dabei sind die Hauptraumblöcke und die Jordanblöcke innerhalb dieser bis auf Reihenfolge eindeutig.
        \cite{Theorem 15.2}
    \end{field}

    \begin{field}
        \setlength{\fboxsep}{0pt}
        \textbf{Jordannormalform}\\
        Sei $f$ ein Endomorphismus eines endlich-dimensionalen Vektorraums.\\
        Wenn \hide{$\chi_f$ in Linearfaktoren zerfällt}, so hat $f$ bezüglich einer geeigneten Basis folgende Gestalt:
        \[_B M_B(f) = \left(\begin{smallmatrix}
        \fbox{\, $H(\dots;a_1)$}\\
        &\fbox{\, $H(\dots;a_2)$}&&\smash{\text{\huge 0}}\\
        &&\ddots\\
        \smash{\text{\huge 0}}&&&\fbox{\, $H(\dots;a_l)$}
        \end{smallmatrix}\right)\]
        Dabei sind die Hauptraumblöcke und die Jordanblöcke innerhalb dieser bis auf Reihenfolge eindeutig.
    \end{field}
    \begin{field}
        \setlength{\fboxsep}{0pt}
        \textbf{Jordannormalform}\\
        Sei $f$ ein Endomorphismus eines endlich-dimensionalen Vektorraums.\\
        Wenn $\chi_f$ in Linearfaktoren zerfällt, so hat $f$ bezüglich einer geeigneten Basis folgende Gestalt:
        \[_B M_B(f) = \left(\begin{smallmatrix}
        \fbox{\, $H(\dots;a_1)$}\\
        &\fbox{\, $H(\dots;a_2)$}&&\smash{\text{\huge 0}}\\
        &&\ddots\\
        \smash{\text{\huge 0}}&&&\fbox{\, $H(\dots;a_l)$}
        \end{smallmatrix}\right)\]
        Dabei sind die Hauptraumblöcke und die Jordanblöcke innerhalb dieser bis auf Reihenfolge eindeutig.
        \cite{Theorem 15.2}
    \end{field}
\end{note}

\begin{note}
    \tags{Satz}
    \xplain{c22833e4-8610-427a-8142-05abf2f142c1}

    \begin{field}
        Für die Jordannormalform von $f$ gilt:
        \begin{itemize}
            \item $a_1,\dots,a_l$ sind \hide{die verschiedenen Eigenwerte von $f$.}
            \item Größe von $H(m_1,\dots,m_k;a)$ ist die algebraische Vielfachheit von $a$.\\
            ($= \max\{r\in \NN \mid (X-a)^r \text{ teilt } \chi_f\}$)
            \item Größe $m$ des größten Jordanblocks $J(m;a)$ zu $a$ ist der Exponent von $(X-a)$ in $\mu_f$.\\
            ($= \max\{r\in \NN \mid (X-a)^r \text{ teilt } \mu_f\}$)
        \end{itemize}
    \end{field}
    \begin{field}
        Für die Jordannormalform von $f$ gilt:
        \begin{itemize}
            \item $a_1,\dots,a_l$ sind die verschiedenen Eigenwerte von $f$.
            \item Größe von $H(m_1,\dots,m_k;a)$ ist die algebraische Vielfachheit von $a$.\\
            ($= \max\{r\in \NN \mid (X-a)^r \text{ teilt } \chi_f\}$)
            \item Größe $m$ des größten Jordanblocks $J(m;a)$ zu $a$ ist der Exponent von $(X-a)$ in $\mu_f$.\\
            ($= \max\{r\in \NN \mid (X-a)^r \text{ teilt } \mu_f\}$)
        \end{itemize}
        \cite{Notiz 15.3}
    \end{field}

    \begin{field}
        Für die Jordannormalform von $f$ gilt:
        \begin{itemize}
            \item $a_1,\dots,a_l$ sind die verschiedenen Eigenwerte von $f$.
            \item Größe von $H(m_1,\dots,m_k;a)$ \hide{ist die algebraische Vielfachheit von $a$.\\
            ($= \max\{r\in \NN \mid (X-a)^r \text{ teilt } \chi_f\}$)}
            \item Größe $m$ des größten Jordanblocks $J(m;a)$ zu $a$ ist der Exponent von $(X-a)$ in $\mu_f$.\\
            ($= \max\{r\in \NN \mid (X-a)^r \text{ teilt } \mu_f\}$)
        \end{itemize}
    \end{field}
    \begin{field}
        Für die Jordannormalform von $f$ gilt:
        \begin{itemize}
            \item $a_1,\dots,a_l$ sind die verschiedenen Eigenwerte von $f$.
            \item Größe von $H(m_1,\dots,m_k;a)$ ist die algebraische Vielfachheit von $a$.\\
            ($= \max\{r\in \NN \mid (X-a)^r \text{ teilt } \chi_f\}$)
            \item Größe $m$ des größten Jordanblocks $J(m;a)$ zu $a$ ist der Exponent von $(X-a)$ in $\mu_f$.\\
            ($= \max\{r\in \NN \mid (X-a)^r \text{ teilt } \mu_f\}$)
        \end{itemize}
        \cite{Notiz 15.3}
    \end{field}

    \begin{field}
        Für die Jordannormalform von $f$ gilt:
        \begin{itemize}
            \item $a_1,\dots,a_l$ sind die verschiedenen Eigenwerte von $f$.
            \item Größe von $H(m_1,\dots,m_k;a)$ ist die algebraische Vielfachheit von $a$.\\
            ($= \max\{r\in \NN \mid (X-a)^r \text{ teilt } \chi_f\}$)
            \item Größe $m$ des größten Jordanblocks $J(m;a)$ zu $a$ \hide{ist der Exponent von $(X-a)$ in $\mu_f$.\\
            ($= \max\{r\in \NN \mid (X-a)^r \text{ teilt } \mu_f\}$)}
        \end{itemize}
    \end{field}
    \begin{field}
        Für die Jordannormalform von $f$ gilt:
        \begin{itemize}
            \item $a_1,\dots,a_l$ sind die verschiedenen Eigenwerte von $f$.
            \item Größe von $H(m_1,\dots,m_k;a)$ ist die algebraische Vielfachheit von $a$.\\
            ($= \max\{r\in \NN \mid (X-a)^r \text{ teilt } \chi_f\}$)
            \item Größe $m$ des größten Jordanblocks $J(m;a)$ zu $a$ ist der Exponent von $(X-a)$ in $\mu_f$.\\
            ($= \max\{r\in \NN \mid (X-a)^r \text{ teilt } \mu_f\}$)
        \end{itemize}
        \cite{Notiz 15.3}
    \end{field}
\end{note}

\begin{note}
    \tags{Def}
    \xplain{f0eae56c-7693-4b1b-b3bf-82466a1966bb}

    \xfield{
    \textbf{Triagonalisierbarkeitskriterium}\\
    Ein Endomorphismus $f$ ist \textbf{triagonalisierbar}, falls \dots
    }
    \begin{field}
        \textbf{Triagonalisierbarkeitskriterium}\\
        Ein Endomorphismus $f$ ist \textbf{triagonalisierbar}, falls eine Basis $B$ existiert, in der $_B M_B(f)$ eine obere Dreiecksmatrix ist.
        \cite{Def. 15.4}
    \end{field}
\end{note}

\begin{note}
    \tags{Def}
    \xplain{24fdceea-c59b-4730-9a1d-506ef970bedf}

    \xfield{
    Sei $a$ ein Eigenwert von $f$, sei $\mu_f = (X-a)^m\cdot P$ mit $P$ teilerfremd zu $(X-a)$.\\
    Der \textbf{Hauptraum} von $f$ zu $a$ ist \dots
    }
    \begin{field}
        Sei $a$ ein Eigenwert von $f$, sei $\mu_f = (X-a)^m\cdot P$ mit $P$ teilerfremd zu $(X-a)$.\\
        Der \textbf{Hauptraum} von $f$ zu $a$ ist
        \[\text{Hau}(f;a) := \ker((f - a\cdot \id)^m)\]
        \cite{Def. 15.7}
    \end{field}
\end{note}

\begin{note}
    \tags{Satz}
    \xplain{73273508-0f38-4c7e-ba80-bf591c7f3800}

    \xfield{
        \textbf{Hauptraumzerlegung}\\
        Zerfällt $\chi_f$ in Linearfaktoren, so \hint{Zerlegung von $V$}\dots
    }
    \begin{field}
        \textbf{Hauptraumzerlegung}\\
        Zerfällt $\chi_f$ in Linearfaktoren, so zerfällt $V$ in die Haupträume:
        \[V = \oplus_{i=1}^l \text{Hau}(f;a_i)\]
        wobei $a_1,\dots,a_l$ die verschiedenen Eigenwerte von $f$ sind.
        \cite{Satz 15.8}
    \end{field}
\end{note}

\begin{note}
    \tags{Satz}
    \xplain{2351ff5a-204c-45bb-8b66-75d2238fbfb3}

    \begin{field}
        \textbf{Eigenschaften der Haupträume}\\
        Sei $a$ Eigenwert von $f$ mit algebraischer Vielfachheit $r$, also $\chi_f = (X-a)^r\cdot P$ und $\mu_f = (X-a)^m\cdot \tilde P$ mit $P$ und $\tilde P$ jeweils teilerfremd zu $(X-a)$.
        \begin{enumerate}[(1)]
            \item \hide{$\text{Hau}(f;a)$ ist $f$-stabil.}
            \item $\chi_f\vert_{\text{Hau}(f;a)} = (-1)^r(X-a)^r$
            \item $\mu_f\vert_{\text{Hau}(f;a)} = (X-a)^m$
            \item $\dim \text{Hau}(f;a) = r$
            \item $\text{Hau}(f;a)= \ker((f- a\cdot \id)^i) \ \forall i \geq m$.
        \end{enumerate}
    \end{field}
    \begin{field}
        \textbf{Eigenschaften der Haupträume}\\
        Sei $a$ Eigenwert von $f$ mit algebraischer Vielfachheit $r$, also $\chi_f = (X-a)^r\cdot P$ und $\mu_f = (X-a)^m\cdot \tilde P$ mit $P$ und $\tilde P$ jeweils teilerfremd zu $(X-a)$.
        \begin{enumerate}[(1)]
            \item $\text{Hau}(f;a)$ ist $f$-stabil.
            \item $\chi_f\vert_{\text{Hau}(f;a)} = (-1)^r(X-a)^r$
            \item $\mu_f\vert_{\text{Hau}(f;a)} = (X-a)^m$
            \item $\dim \text{Hau}(f;a) = r$
            \item $\text{Hau}(f;a)= \ker((f- a\cdot \id)^i) \ \forall i \geq m$.
        \end{enumerate}
        \cite{Satz 15.9}
    \end{field}

    \begin{field}
        \textbf{Eigenschaften der Haupträume}\\
        Sei $a$ Eigenwert von $f$ mit algebraischer Vielfachheit $r$, also $\chi_f = (X-a)^r\cdot P$ und $\mu_f = (X-a)^m\cdot \tilde P$ mit $P$ und $\tilde P$ jeweils teilerfremd zu $(X-a)$.
        \begin{enumerate}[(1)]
            \item $\text{Hau}(f;a)$ ist $f$-stabil.
            \item $\chi_f\vert_{\text{Hau}(f;a)} = \phantom{(-1)^r(X-a)^r}$
            \item $\mu_f\vert_{\text{Hau}(f;a)} = (X-a)^m$
            \item $\dim \text{Hau}(f;a) = r$
            \item $\text{Hau}(f;a)= \ker((f- a\cdot \id)^i) \ \forall i \geq m$.
        \end{enumerate}
    \end{field}
    \begin{field}
        \textbf{Eigenschaften der Haupträume}\\
        Sei $a$ Eigenwert von $f$ mit algebraischer Vielfachheit $r$, also $\chi_f = (X-a)^r\cdot P$ und $\mu_f = (X-a)^m\cdot \tilde P$ mit $P$ und $\tilde P$ jeweils teilerfremd zu $(X-a)$.
        \begin{enumerate}[(1)]
            \item $\text{Hau}(f;a)$ ist $f$-stabil.
            \item $\chi_f\vert_{\text{Hau}(f;a)} = (-1)^r(X-a)^r$
            \item $\mu_f\vert_{\text{Hau}(f;a)} = (X-a)^m$
            \item $\dim \text{Hau}(f;a) = r$
            \item $\text{Hau}(f;a)= \ker((f- a\cdot \id)^i) \ \forall i \geq m$.
        \end{enumerate}
        \cite{Satz 15.9}
    \end{field}

    \begin{field}
        \textbf{Eigenschaften der Haupträume}\\
        Sei $a$ Eigenwert von $f$ mit algebraischer Vielfachheit $r$, also $\chi_f = (X-a)^r\cdot P$ und $\mu_f = (X-a)^m\cdot \tilde P$ mit $P$ und $\tilde P$ jeweils teilerfremd zu $(X-a)$.
        \begin{enumerate}[(1)]
            \item $\text{Hau}(f;a)$ ist $f$-stabil.
            \item $\chi_f\vert_{\text{Hau}(f;a)} = (-1)^r(X-a)^r$
            \item $\mu_f\vert_{\text{Hau}(f;a)} = \phantom{(X-a)^m}$
            \item $\dim \text{Hau}(f;a) = r$
            \item $\text{Hau}(f;a)= \ker((f- a\cdot \id)^i) \ \forall i \geq m$.
        \end{enumerate}
    \end{field}
    \begin{field}
        \textbf{Eigenschaften der Haupträume}\\
        Sei $a$ Eigenwert von $f$ mit algebraischer Vielfachheit $r$, also $\chi_f = (X-a)^r\cdot P$ und $\mu_f = (X-a)^m\cdot \tilde P$ mit $P$ und $\tilde P$ jeweils teilerfremd zu $(X-a)$.
        \begin{enumerate}[(1)]
            \item $\text{Hau}(f;a)$ ist $f$-stabil.
            \item $\chi_f\vert_{\text{Hau}(f;a)} = (-1)^r(X-a)^r$
            \item $\mu_f\vert_{\text{Hau}(f;a)} = (X-a)^m$
            \item $\dim \text{Hau}(f;a) = r$
            \item $\text{Hau}(f;a)= \ker((f- a\cdot \id)^i) \ \forall i \geq m$.
        \end{enumerate}
        \cite{Satz 15.9}
    \end{field}

    \begin{field}
        \textbf{Eigenschaften der Haupträume}\\
        Sei $a$ Eigenwert von $f$ mit algebraischer Vielfachheit $r$, also $\chi_f = (X-a)^r\cdot P$ und $\mu_f = (X-a)^m\cdot \tilde P$ mit $P$ und $\tilde P$ jeweils teilerfremd zu $(X-a)$.
        \begin{enumerate}[(1)]
            \item $\text{Hau}(f;a)$ ist $f$-stabil.
            \item $\chi_f\vert_{\text{Hau}(f;a)} = (-1)^r(X-a)^r$
            \item $\mu_f\vert_{\text{Hau}(f;a)} = (X-a)^m$
            \item $\dim \text{Hau}(f;a) = \phantom{r}$
            \item $\text{Hau}(f;a)= \ker((f- a\cdot \id)^i) \ \forall i \geq m$.
        \end{enumerate}
    \end{field}
    \begin{field}
        \textbf{Eigenschaften der Haupträume}\\
        Sei $a$ Eigenwert von $f$ mit algebraischer Vielfachheit $r$, also $\chi_f = (X-a)^r\cdot P$ und $\mu_f = (X-a)^m\cdot \tilde P$ mit $P$ und $\tilde P$ jeweils teilerfremd zu $(X-a)$.
        \begin{enumerate}[(1)]
            \item $\text{Hau}(f;a)$ ist $f$-stabil.
            \item $\chi_f\vert_{\text{Hau}(f;a)} = (-1)^r(X-a)^r$
            \item $\mu_f\vert_{\text{Hau}(f;a)} = (X-a)^m$
            \item $\dim \text{Hau}(f;a) = r$
            \item $\text{Hau}(f;a)= \ker((f- a\cdot \id)^i) \ \forall i \geq m$.
        \end{enumerate}
        \cite{Satz 15.9}
    \end{field}
\end{note}

\begin{note}
    \tags{Satz}
    \xplain{5416b650-4583-4a20-90b5-6c32a203b282}

    \begin{field}
        \textbf{Eigenschaften der Haupträume}\\
        Sei $a$ Eigenwert von $f$ mit algebraischer Vielfachheit $r$, also $\chi_f = (X-a)^r\cdot P$ und $\mu_f = (X-a)^m\cdot \tilde P$ mit $P$ und $\tilde P$ jeweils teilerfremd zu $(X-a)$.
        \begin{enumerate}[(1)]
            \item $\text{Hau}(f;a)$ ist $f$-stabil.
            \item $\chi_f\vert_{\text{Hau}(f;a)} = (-1)^r(X-a)^r$
            \item $\mu_f\vert_{\text{Hau}(f;a)} = (X-a)^m$
            \item $\dim \text{Hau}(f;a) = r$
            \item $\text{Hau}(f;a)= \phantom{\ker((f- a\cdot \id)^i) \ \forall i \geq m}$.
        \end{enumerate}
    \end{field}
    \begin{field}
        \textbf{Eigenschaften der Haupträume}\\
        Sei $a$ Eigenwert von $f$ mit algebraischer Vielfachheit $r$, also $\chi_f = (X-a)^r\cdot P$ und $\mu_f = (X-a)^m\cdot \tilde P$ mit $P$ und $\tilde P$ jeweils teilerfremd zu $(X-a)$.
        \begin{enumerate}[(1)]
            \item $\text{Hau}(f;a)$ ist $f$-stabil.
            \item $\chi_f\vert_{\text{Hau}(f;a)} = (-1)^r(X-a)^r$
            \item $\mu_f\vert_{\text{Hau}(f;a)} = (X-a)^m$
            \item $\dim \text{Hau}(f;a) = r$
            \item $\text{Hau}(f;a)= \ker((f- a\cdot \id)^i) \ \forall i \geq m$.
        \end{enumerate}
        \cite{Satz 15.9}
    \end{field}
\end{note}

%%%%%%%%% Vorlesung 14 %%%%%%%%%%%%

\begin{note}
    \tags{Def}
    \xplain{32fa39ce-09db-4219-a5c1-6c2e711728c6}

    \xfield{
    Ein Endomorphismus $g$ ist \textbf{nilpotent}, wenn \dots
    }
    \begin{field}
        Ein Endomorphismus $g$ ist \textbf{nilpotent}, wenn $g^k = 0$ für ein $k\in\NN$.
        \cite{Def. 15.10}
    \end{field}
\end{note}

\begin{note}
    \tags{Satz}
    \xplain{36322b72-9b20-482b-bf70-882813cdb5db}

    \begin{field}
        \small
        Für einen (endlich-dimenionalen) Vektorraum $V$ mit Untervektorräumen $U_1,\dots,U_k$ sind äquivalent:
        \begin{enumerate}[(1)]
            \item \hide{$V = \oplus_{i=1}^k U_i$}
            \item $V$ hat eine Basis der Form \[(\vec{u}_1^{(1)},\dots,\vec{u}_{m_1}^{(1)},\vec{u}_1^{(2)},\dots,\vec{u}_{m_2}^{(2)},\dots,\vec{u}_1^{(k)},\dots,\vec{u}_{m_k}^{(k)})\]
            derart, dass $(\vec{u}_1^{(i)}, \dots, \vec{u}_{m_i}^{(i)})$ Basis von $U_i$ ist.
            \item Für beliebige Basen $(\vec{u}_1^{(i)}, \dots, \vec{u}_{m_i}^{(i)})$ von $U_i$ ist
             \[(\vec{u}_1^{(1)},\dots,\vec{u}_{m_1}^{(1)},\vec{u}_1^{(2)},\dots,\vec{u}_{m_2}^{(2)},\dots,\vec{u}_1^{(k)},\dots,\vec{u}_{m_k}^{(k)})\]
             eine Basis von $V$.
        \end{enumerate}
    \end{field}
    \begin{field}
        \small
        Für einen (endlich-dimenionalen) Vektorraum $V$ mit Untervektorräumen $U_1,\dots,U_k$ sind äquivalent:
        \begin{enumerate}[(1)]
            \item $V = \oplus_{i=1}^k U_i$
            \item $V$ hat eine Basis der Form \[(\vec{u}_1^{(1)},\dots,\vec{u}_{m_1}^{(1)},\vec{u}_1^{(2)},\dots,\vec{u}_{m_2}^{(2)},\dots,\vec{u}_1^{(k)},\dots,\vec{u}_{m_k}^{(k)})\]
            derart, dass $(\vec{u}_1^{(i)}, \dots, \vec{u}_{m_i}^{(i)})$ Basis von $U_i$ ist.
            \item Für beliebige Basen $(\vec{u}_1^{(i)}, \dots, \vec{u}_{m_i}^{(i)})$ von $U_i$ ist
             \[(\vec{u}_1^{(1)},\dots,\vec{u}_{m_1}^{(1)},\vec{u}_1^{(2)},\dots,\vec{u}_{m_2}^{(2)},\dots,\vec{u}_1^{(k)},\dots,\vec{u}_{m_k}^{(k)})\]
             eine Basis von $V$.
        \end{enumerate}
        \cite{Notiz 15.13}
    \end{field}

    \begin{field}
        \small
        Für einen (endlich-dimenionalen) Vektorraum $V$ mit Untervektorräumen $U_1,\dots,U_k$ sind äquivalent:
        \begin{enumerate}[(1)]
            \item $V = \oplus_{i=1}^k U_i$
            \item \hide{$V$ hat eine Basis der Form \[(\vec{u}_1^{(1)},\dots,\vec{u}_{m_1}^{(1)},\vec{u}_1^{(2)},\dots,\vec{u}_{m_2}^{(2)},\dots,\vec{u}_1^{(k)},\dots,\vec{u}_{m_k}^{(k)})\]
            derart, dass $(\vec{u}_1^{(i)}, \dots, \vec{u}_{m_i}^{(i)})$ Basis von $U_i$ ist.}
            \item Für beliebige Basen $(\vec{u}_1^{(i)}, \dots, \vec{u}_{m_i}^{(i)})$ von $U_i$ ist
             \[(\vec{u}_1^{(1)},\dots,\vec{u}_{m_1}^{(1)},\vec{u}_1^{(2)},\dots,\vec{u}_{m_2}^{(2)},\dots,\vec{u}_1^{(k)},\dots,\vec{u}_{m_k}^{(k)})\]
             eine Basis von $V$.
        \end{enumerate}
    \end{field}
    \begin{field}
        \small
        Für einen (endlich-dimenionalen) Vektorraum $V$ mit Untervektorräumen $U_1,\dots,U_k$ sind äquivalent:
        \begin{enumerate}[(1)]
            \item $V = \oplus_{i=1}^k U_i$
            \item $V$ hat eine Basis der Form \[(\vec{u}_1^{(1)},\dots,\vec{u}_{m_1}^{(1)},\vec{u}_1^{(2)},\dots,\vec{u}_{m_2}^{(2)},\dots,\vec{u}_1^{(k)},\dots,\vec{u}_{m_k}^{(k)})\]
            derart, dass $(\vec{u}_1^{(i)}, \dots, \vec{u}_{m_i}^{(i)})$ Basis von $U_i$ ist.
            \item Für beliebige Basen $(\vec{u}_1^{(i)}, \dots, \vec{u}_{m_i}^{(i)})$ von $U_i$ ist
             \[(\vec{u}_1^{(1)},\dots,\vec{u}_{m_1}^{(1)},\vec{u}_1^{(2)},\dots,\vec{u}_{m_2}^{(2)},\dots,\vec{u}_1^{(k)},\dots,\vec{u}_{m_k}^{(k)})\]
             eine Basis von $V$.
        \end{enumerate}
        \cite{Notiz 15.13}
    \end{field}

    \begin{field}
        \small
        Für einen (endlich-dimenionalen) Vektorraum $V$ mit Untervektorräumen $U_1,\dots,U_k$ sind äquivalent:
        \begin{enumerate}[(1)]
            \item $V = \oplus_{i=1}^k U_i$
            \item $V$ hat eine Basis der Form \[(\vec{u}_1^{(1)},\dots,\vec{u}_{m_1}^{(1)},\vec{u}_1^{(2)},\dots,\vec{u}_{m_2}^{(2)},\dots,\vec{u}_1^{(k)},\dots,\vec{u}_{m_k}^{(k)})\]
            derart, dass $(\vec{u}_1^{(i)}, \dots, \vec{u}_{m_i}^{(i)})$ Basis von $U_i$ ist.
            \item \hide{Für beliebige Basen $(\vec{u}_1^{(i)}, \dots, \vec{u}_{m_i}^{(i)})$ von $U_i$ ist
             \[(\vec{u}_1^{(1)},\dots,\vec{u}_{m_1}^{(1)},\vec{u}_1^{(2)},\dots,\vec{u}_{m_2}^{(2)},\dots,\vec{u}_1^{(k)},\dots,\vec{u}_{m_k}^{(k)})\]
             eine Basis von $V$.}
        \end{enumerate}
    \end{field}
    \begin{field}
        \small
        Für einen (endlich-dimenionalen) Vektorraum $V$ mit Untervektorräumen $U_1,\dots,U_k$ sind äquivalent:
        \begin{enumerate}[(1)]
            \item $V = \oplus_{i=1}^k U_i$
            \item $V$ hat eine Basis der Form \[(\vec{u}_1^{(1)},\dots,\vec{u}_{m_1}^{(1)},\vec{u}_1^{(2)},\dots,\vec{u}_{m_2}^{(2)},\dots,\vec{u}_1^{(k)},\dots,\vec{u}_{m_k}^{(k)})\]
            derart, dass $(\vec{u}_1^{(i)}, \dots, \vec{u}_{m_i}^{(i)})$ Basis von $U_i$ ist.
            \item Für beliebige Basen $(\vec{u}_1^{(i)}, \dots, \vec{u}_{m_i}^{(i)})$ von $U_i$ ist
             \[(\vec{u}_1^{(1)},\dots,\vec{u}_{m_1}^{(1)},\vec{u}_1^{(2)},\dots,\vec{u}_{m_2}^{(2)},\dots,\vec{u}_1^{(k)},\dots,\vec{u}_{m_k}^{(k)})\]
             eine Basis von $V$.
        \end{enumerate}
        \cite{Notiz 15.13}
    \end{field}
\end{note}

\begin{note}
    \tags{Def}
    \xplain{0e4987e4-ed28-4005-83aa-1d66ccedb48b}

    \xfield{
    Ein \textbf{komplementärer Untervektorraum} zu einem Untervektorraum $W\subseteq V$ ist ein Untervektorraum $U$\dots
    }
    \begin{field}
        Ein \textbf{komplementärer Untervektorraum} zu einem Untervektorraum $W\subseteq V$ ist ein Untervektorraum $U\subseteq V$ mit
        \[V = W \oplus U\]
        \cite{Def. 14.15}
    \end{field}
\end{note}

\begin{note}
    \tags{Satz}
    \xplain{eba13948-a4e5-4157-ad25-ff5e44396703}

    \xfield{
    \textbf{Jordannormalform im nilpotenten Fall}\\
    Zu jedem nilpotenten Endomorphismus $g$ existiert \dots
    }
    \begin{field}
    \small
    \textbf{Jordannormalform im nilpotenten Fall}\\
    Zu jedem nilpotenten Endomorphismus $g$ existiert eine \emph{Jordanbasis}, also eine Basis $B$, in der $_B M_B(g)$ JNF hat
    \begin{align*}
        _B M_B(g) &= H(m_1,\dots,m_k;0)\\
            &= \left(\begin{smallmatrix}
                J(m_1;0)\\
                &J(m_2;0)&&\smash{\text{\LARGE 0}}\\\\
                &&\ddots\\
                \smash{\text{\LARGE 0}}&&&J(m_3;0)
            \end{smallmatrix}\right)
    \end{align*}
    mit $J(m;0) := \left(\begin{smallmatrix}
        0&1&&\smash{\text{\Large 0}}\\
        &\ddots&\ddots\\
        &&\ddots&1\\
        \smash{\text{\Large 0}}&&&0
    \end{smallmatrix}\right)$
    \cite{Satz 15.16}
    \end{field}
\end{note}

%%%%%%%%% Vorlesung 15 %%%%%%%%%%%%

%%%%%%%%% Vorlesung 16 %%%%%%%%%%%%


\end{document}
