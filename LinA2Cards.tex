% -*- coding: utf-8-unix -*-
%%%%%%%%%%%%%%%%%%%%%%%%%%%%%%%%%%%%%%%%%%%%%%%%%%%%
% The first part of the header needs to be copied
%       into the note options in Anki.
%%%%%%%%%%%%%%%%%%%%%%%%%%%%%%%%%%%%%%%%%%%%%%%%%%%%

% layout in Anki:
\documentclass[11pt]{article}
\usepackage[a4paper]{geometry}
\geometry{paperwidth=.5\paperwidth,paperheight=100in,left=2em,right=2em,bottom=1em,top=2em}
\pagestyle{empty}
\setlength{\parindent}{0in}
 
% hyphenation:
\usepackage[ngerman]{babel}

% encoding:
\usepackage[T1]{fontenc}
\usepackage[utf8]{inputenc}
\usepackage{lmodern}

% packages:
\usepackage{
  % maths:
  amssymb,
  amsmath,
  mathtools,  % (corrects some bugs in ams-packages)
  bm,         % (bold maths symbols)
  calc,
  nicefrac,
  stmaryrd,   % (symbols)  
  % microtype, don't use this:  
  %   line breaks will differ between latex & pdflatex output
  eqparbox,
  paralist,
  xcolor,
  bbm         % mathbbm = math-blackboard-bold
} 
\renewcommand{\cite}[1]{\par\bigskip\hfill{\color{gray}\tiny\(\to\) #1}}
% Diagrams:
\usepackage[all]{xy}

% Blackboard:
\newcommand{\CC}{\mathbb{C}}
\newcommand{\FF}{\mathbb{F}}
\newcommand{\NN}{\mathbb{N}}
\newcommand{\QQ}{\mathbb{Q}}
\newcommand{\RR}{\mathbb{R}}
\newcommand{\ZZ}{\mathbb{Z}}
\newcommand{\ve}{\underline}

% (Redefined) symobls:
\newcommand*{\abs}[1]{\left\vert#1\right\vert}
\newcommand*{\norm}[1]{\left\|#1\right\|}
\newcommand*{\scprod}[2]{\langle #1, #2\rangle} %Skalarprodukt
\renewcommand{\phi}{\varphi}
\renewcommand{\epsilon}{\varepsilon}
%\renewcommand{\setminus}{\!-\!}
\renewcommand{\leq}{\leqslant}
\renewcommand{\geq}{\geqslant}
\newcommand{\tsmash}{\wedge}
\newcommand{\twedge}{\vee}
\newcommand{\from}{\leftarrow}
\newcommand{\noloc}{:\!}
% Special maps:
\newcommand{\id}{\mathrm{id}}

% Linear algebra:
\renewcommand{\vec}[1]{\mathbf{#1}}
\newcommand{\Potenz}{\mathfrak P}
%\newcommand{\im}{\mathrm{im}}
\newcommand{\hull}[1]{\langle #1 \rangle}
\newcommand{\Lraum}{\mathcal L}

% Categories:
\DeclareMathOperator{\Ar}{Ar}
\usepackage{bm}
\newcommand{\cat}[1]{\bm{\mathsf{#1}}}
\newcommand{\iTop}{\underline{\smash{\mathsf{Top}}}}  % internal hom in Top
\newcommand{\iAut}{\underline{\smash{\mathsf{Aut}}}}  
\newcommand{\Aut}{\mathrm{Aut}}
\newcommand{\Mor}{\mathrm{Mor}}
\newcommand{\Hom}{\mathrm{Hom}}
\DeclareMathOperator{\signum}{sgn}
\DeclareMathOperator{\rank}{rk}
\DeclareMathOperator{\im}{im}
\DeclareMathOperator{\Abb}{Abb}
\DeclareMathOperator{\Mat}{Mat}



\mathchardef\mhyphen="2D
\newcommand*{\factor}[2]{\left.\raisebox{.1em}{\ensuremath{#1}}\middle/\raisebox{-.1em}{\ensuremath{#2}}\right.}

% arrows:
\newcommand*{\openhookrightarrow}{\mathrel{\ooalign{$\hookrightarrow$\cr\hidewidth\hbox{$\circ\,\,$}\cr}}}
\newcommand*{\closedhookrightarrow}{\mathrel{\ooalign{$\hookrightarrow$\cr\hidewidth\raise0.1ex\hbox{$\rule{0.5pt}{1ex}\;\;$}\cr}}}
\newcommand*{\openhookleftarrow}{\mathrel{\ooalign{$\hookleftarrow$\cr\hidewidth\hbox{$\circ\;$}\cr}}}
\newcommand*{\closedhookleftarrow}{\mathrel{\ooalign{$\hookleftarrow$\cr\hidewidth\raise0.1ex\hbox{$\rule{0.5pt}{1ex}\,\;$}\cr}}}
\newcommand*{\leadsleadsto}{\mathrel{\substack{\leadsto\\[-1em]\leadsto}}}


% text positioning in maths:
\newcommand{\ctext}[2]{\text{\parbox{#1}{\centering #2}}}
\newcommand{\ltext}[2]{\text{\parbox{#1}{\raggedright#2}}}
\newcommand{\rtext}[2]{\text{\parbox{#1}{\raggedleft#2}}}
\newcommand{\cttext}[2]{\text{\parbox[t]{#1}{\centering #2}}}
\newcommand{\lttext}[2]{\text{\parbox[t]{#1}{\raggedright#2}}}
\newcommand{\rttext}[2]{\text{\parbox[t]{#1}{\raggedleft#2}}}
% Hide
\newcommand{\hide}[1]{\parbox{0cm}{\raisebox{-7pt}[0cm]{\dots}}\color{white}#1\color{black}}
\newcommand{\hint}[1]{{\color{lightgray}(#1)}}
\let\olddots\dots
\renewcommand{\dots}{\,\olddots\,}

% bold text sans serif
\DeclareTextFontCommand{\textbf}{\bfseries\sffamily} 
\hyphenation{
  Vektor-raum 
  Vektor-raums
  Unter-vektor-raum
  Unter-vektor-raums
  Ortho-normal-basis
  Skalar-produkt
  Standard-skalar-produkt
}
%%%%%%%%%%%%%%%%%%%%%%%%%%%%%%%%%%%%%%%%%%%%%%%%%%%%
% Following part of header NOT to be copied into
%            the note options in Anki.
%          ! Anki will throw an errow !
%%%%%%%%%%%%%%%%%%%%%%%%%%%%%%%%%%%%%%%%%%%%%%%%%%%%%
%
%  pdf layout:
%
\geometry{paperheight=74.25mm}
\usepackage{pgfpages}
\pagestyle{empty}
\pgfpagesuselayout{8 on 1}[a4paper,border shrink=0cm]
\makeatletter
\@tempcnta=1\relax
\loop\ifnum\@tempcnta<9\relax
\pgf@pset{\the\@tempcnta}{bordercode}{\pgfusepath{stroke}}
\advance\@tempcnta by 1\relax
\repeat
\makeatother
% 
%  notes, fields, tags:
%
\newcommand{\xfield}[1]{
        #1\par
        \vfill
        {\tiny\texttt{\parbox[t]{\textwidth}{\localtag\\\globaltag\hfill\uuid}}}
        \newpage}
\newenvironment{field}{}{\newpage}
\newif\ifnote
\newenvironment{note}{\notetrue}{\notefalse}
\newcommand{\localtag}{}
\newcommand{\globaltag}{}
\newcommand{\uuid}{}
\newcommand{\tags}[1]{
    \ifnote 
        \renewcommand{\localtag}{#1}
    \else
        \renewcommand{\globaltag}{#1}
    \fi 
    }
\newcommand{\xplain}[1]{\renewcommand{\uuid}{#1}}
%
%%%%%%%%%%%%%%%%%%%%%%%%%%%%%%%%%%%%%%%%%%%%%%%%%%%%
% The following line again needs to be copied 
% into Anki:
\begin{document}
%%%%%%%%%%%%%%%%%%%%%%%%%%%%%%%%%%%%%%%%%%%%%%%%%%%%

\tags{LinA-II-10-Skalarprodukte}
%%%%%%%%% Vorlesung 1 %%%%%%%%%%%%
\begin{note}
    \xplain{6fc8a1b6-b98b-11ec-8422-0242ac120002}
    \tags{Def}
    \begin{field} %Frage 1
    Sei \(K\) ein Körper, \(V\) ein \(K\)-VR. Eine \textbf{Bilinearform} auf \(V\) ist eine (bilineare) Abbildung  \(\beta\colon V\times V \to K\), für die gilt:
    \begin{itemize}
        \item[(1)] \hide{\(\beta(\vec{v_1} + \vec{v_2}, \vec{w}) = \beta(\vec{v_1}, \vec{w}) + \beta(\vec{v_2}, \vec{w})\) für alle \(\vec{v_1, v_2, w} \in V\)}
        \item[(2)] \(\beta(s\cdot\vec{v},\vec{w}) = s\cdot\beta(\vec{v},\vec{w})\) für alle \(s\in K\),\(\vec{v},\vec{w}\in V\)
        \item[(3)]\(\beta(\vec{v}, \vec{w_1} + \vec{w_2}) = \beta(\vec{v}, \vec{w_1}) + \beta(\vec{v}, \vec{w_2})\) für alle \(\vec{v, w_1, w_2} \in V\)
        \item[(4)]\(\beta(\vec{v},s\cdot\vec{w}) = s\cdot\beta(\vec{v},\vec{w})\) für alle \(s\in K\),\(\vec{v},\vec{w}\in V\)
    \end{itemize}
    \end{field}
    \begin{field} %Antwort 1
    Sei \(K\) ein Körper, \(V\) ein \(K\)-VR. Eine \textbf{Bilinearform} auf \(V\) ist eine (bilineare) Abbildung  \(\beta\colon V\times V \to K\), für die gilt:
    \begin{itemize}
        \item[(1)] \(\beta(\vec{v_1} + \vec{v_2}, \vec{w}) = \beta(\vec{v_1}, \vec{w}) + \beta(\vec{v_2}, \vec{w})\) für alle \(\vec{v_1, v_2, w} \in V\)
        \item[(2)] \(\beta(s\cdot\vec{v},\vec{w}) = s\cdot\beta(\vec{v},\vec{w})\) für alle \(s\in K\),\(\vec{v},\vec{w}\in V\)
        \item[(3)]\(\beta(\vec{v}, \vec{w_1} + \vec{w_2}) = \beta(\vec{v}, \vec{w_1}) + \beta(\vec{v}, \vec{w_2})\) für alle \(\vec{v, w_1, w_2} \in V\)
        \item[(4)]\(\beta(\vec{v},s\cdot\vec{w}) = s\cdot\beta(\vec{v},\vec{w})\) für alle \(s\in K\),\(\vec{v},\vec{w}\in V\)
    \end{itemize}
    \cite{Def. ~10.1}
    \end{field}
    \begin{field} %Frage 2
    Sei \(K\) ein Körper, \(V\) ein \(K\)-VR. Eine \textbf{Bilinearform} auf \(V\) ist eine (bilineare) Abbildung  \(\beta\colon V\times V \to K\), für die gilt:
    \begin{itemize}
        \item[(1)] \(\beta(\vec{v_1} + \vec{v_2}, \vec{w}) = \beta(\vec{v_1}, \vec{w}) + \beta(\vec{v_2}, \vec{w})\) für alle \(\vec{v_1, v_2, w} \in V\)
        \item[(2)] \hide{\(\beta(s\cdot\vec{v},\vec{w}) = s\cdot\beta(\vec{v},\vec{w})\) für alle \(s\in K\),\(\vec{v},\vec{w}\in V\)}
        \item[(3)]\(\beta(\vec{v}, \vec{w_1} + \vec{w_2}) = \beta(\vec{v}, \vec{w_1}) + \beta(\vec{v}, \vec{w_2})\) für alle \(\vec{v, w_1, w_2} \in V\)
        \item[(4)]\(\beta(\vec{v},s\cdot\vec{w}) = s\cdot\beta(\vec{v},\vec{w})\) für alle \(s\in K\),\(\vec{v},\vec{w}\in V\)
    \end{itemize}
    \end{field}
    \begin{field} %Antwort 2
    Sei \(K\) ein Körper, \(V\) ein \(K\)-VR. Eine \textbf{Bilinearform} auf \(V\) ist eine (bilineare) Abbildung  \(\beta\colon V\times V \to K\), für die gilt:
    \begin{itemize}
        \item[(1)] \(\beta(\vec{v_1} + \vec{v_2}, \vec{w}) = \beta(\vec{v_1}, \vec{w}) + \beta(\vec{v_2}, \vec{w})\) für alle \(\vec{v_1, v_2, w} \in V\)
        \item[(2)] \(\beta(s\cdot\vec{v},\vec{w}) = s\cdot\beta(\vec{v},\vec{w})\) für alle \(s\in K\),\(\vec{v},\vec{w}\in V\)
        \item[(3)]\(\beta(\vec{v}, \vec{w_1} + \vec{w_2}) = \beta(\vec{v}, \vec{w_1}) + \beta(\vec{v}, \vec{w_2})\) für alle \(\vec{v, w_1, w_2} \in V\)
        \item[(4)]\(\beta(\vec{v},s\cdot\vec{w}) = s\cdot\beta(\vec{v},\vec{w})\) für alle \(s\in K\),\(\vec{v},\vec{w}\in V\)
    \end{itemize}
    \cite{Def. ~10.1}
    \end{field}
    \begin{field} %Frage 3
    Sei \(K\) ein Körper, \(V\) ein \(K\)-VR. Eine \textbf{Bilinearform} auf \(V\) ist eine (bilineare) Abbildung  \(\beta\colon V\times V \to K\), für die gilt:
    \begin{itemize}
        \item[(1)] \(\beta(\vec{v_1} + \vec{v_2}, \vec{w}) = \beta(\vec{v_1}, \vec{w}) + \beta(\vec{v_2}, \vec{w})\) für alle \(\vec{v_1, v_2, w} \in V\)
        \item[(2)] \(\beta(s\cdot\vec{v},\vec{w}) = s\cdot\beta(\vec{v},\vec{w})\) für alle \(s\in K\),\(\vec{v},\vec{w}\in V\)
        \item[(3)]\hide{\(\beta(\vec{v}, \vec{w_1} + \vec{w_2}) = \beta(\vec{v}, \vec{w_1}) + \beta(\vec{v}, \vec{w_2})\) für alle \(\vec{v, w_1, w_2} \in V\)}
        \item[(4)]\(\beta(\vec{v},s\cdot\vec{w}) = s\cdot\beta(\vec{v},\vec{w})\) für alle \(s\in K\),\(\vec{v},\vec{w}\in V\)
    \end{itemize}
    \end{field}
    \begin{field} %Antwort 3
    Sei \(K\) ein Körper, \(V\) ein \(K\)-VR. Eine \textbf{Bilinearform} auf \(V\) ist eine (bilineare) Abbildung  \(\beta\colon V\times V \to K\), für die gilt:
    \begin{itemize}
        \item[(1)] \(\beta(\vec{v_1} + \vec{v_2}, \vec{w}) = \beta(\vec{v_1}, \vec{w}) + \beta(\vec{v_2}, \vec{w})\) für alle \(\vec{v_1, v_2, w} \in V\)
        \item[(2)] \(\beta(s\cdot\vec{v},\vec{w}) = s\cdot\beta(\vec{v},\vec{w})\) für alle \(s\in K\),\(\vec{v},\vec{w}\in V\)
        \item[(3)]\(\beta(\vec{v}, \vec{w_1} + \vec{w_2}) = \beta(\vec{v}, \vec{w_1}) + \beta(\vec{v}, \vec{w_2})\) für alle \(\vec{v, w_1, w_2} \in V\)
        \item[(4)]\(\beta(\vec{v},s\cdot\vec{w}) = s\cdot\beta(\vec{v},\vec{w})\) für alle \(s\in K\),\(\vec{v},\vec{w}\in V\)
    \end{itemize}
    \cite{Def. ~10.1}
    \end{field}
    \begin{field} %Frage 4
    Sei \(K\) ein Körper, \(V\) ein \(K\)-VR. Eine \textbf{Bilinearform} auf \(V\) ist eine (bilineare) Abbildung  \(\beta\colon V\times V \to K\), für die gilt:
    \begin{itemize}
        \item[(1)] \(\beta(\vec{v_1} + \vec{v_2}, \vec{w}) = \beta(\vec{v_1}, \vec{w}) + \beta(\vec{v_2}, \vec{w})\) für alle \(\vec{v_1, v_2, w} \in V\)
        \item[(2)] \(\beta(s\cdot\vec{v},\vec{w}) = s\cdot\beta(\vec{v},\vec{w})\) für alle \(s\in K\),\(\vec{v},\vec{w}\in V\)
        \item[(3)]\(\beta(\vec{v}, \vec{w_1} + \vec{w_2}) = \beta(\vec{v}, \vec{w_1}) + \beta(\vec{v}, \vec{w_2})\) für alle \(\vec{v, w_1, w_2} \in V\)
        \item[(4)]\hide{\(\beta(\vec{v},s\cdot\vec{w}) = s\cdot\beta(\vec{v},\vec{w})\) für alle \(s\in K\),\(\vec{v},\vec{w}\in V\)}
    \end{itemize}
    \end{field}
    \begin{field}%Antwort 4
    Sei \(K\) ein Körper, \(V\) ein \(K\)-VR. Eine \textbf{Bilinearform} auf \(V\) ist eine (bilineare) Abbildung  \(\beta\colon V\times V \to K\), für die gilt:
    \begin{itemize}
        \item[(1)] \(\beta(\vec{v_1} + \vec{v_2}, \vec{w}) = \beta(\vec{v_1}, \vec{w}) + \beta(\vec{v_2}, \vec{w})\) für alle \(\vec{v_1, v_2, w} \in V\)
        \item[(2)] \(\beta(s\cdot\vec{v},\vec{w}) = s\cdot\beta(\vec{v},\vec{w})\) für alle \(s\in K\),\(\vec{v},\vec{w}\in V\)
        \item[(3)]\(\beta(\vec{v}, \vec{w_1} + \vec{w_2}) = \beta(\vec{v}, \vec{w_1}) + \beta(\vec{v}, \vec{w_2})\) für alle \(\vec{v, w_1, w_2} \in V\)
        \item[(4)]\(\beta(\vec{v},s\cdot\vec{w}) = s\cdot\beta(\vec{v},\vec{w})\) für alle \(s\in K\),\(\vec{v},\vec{w}\in V\)
    \end{itemize}
    \cite{Def. ~10.1}
    \end{field}
\end{note}

\begin{note}
    \tags{Satz}
    \xplain{d607e83e-b98f-11ec-8422-0242ac120002}
    \begin{field}
    Jede Bilinearform \(\beta\) auf \(K^n\) liefert eine Matrix \hide{\(M(\beta)\in \Mat_K(n\times n)\) der Gestalt}
    \end{field}
    \begin{field}
    Jede Bilinearform \(\beta\) auf \(K^n\) liefert eine Matrix \(M(\beta)\in \Mat_K(n\times n)\) der Gestalt
    \[M(\beta)\colon = \beta(\vec{e}_i, \vec{e}_j)_{ij}\] \cite{Satz ~10.2}
    \end{field}
    \begin{field}
    Jede Matrix \(A\in\Mat_K(n\times n)\) liefert eine \hide{Bilinearform auf \(K^n\)} wie folgt:
    \end{field}
    \begin{field}
    Jede Matrix \(A\in\Mat_K(n\times n)\) liefert eine Bilinearform auf \(K^n\) wie folgt:
    \begin{align*}
    \beta_A: K^n\times K^n &\longrightarrow K \\
    (\vec{v}, \vec{w}) &\mapsto \vec{v}^T A \vec{w}
    \end{align*}
    \cite{Satz ~10.2}
    \end{field}
    \begin{field}
    Die Menge der Bilinearformen auf \(K^n\) und die Menge der \(n\times n\) Matrizen über \(K\) sind \hide{isomorph}.
    \end{field}
    \begin{field}
    Die Menge der Bilinearformen auf \(K^n\) und die Menge der \(n\times n\) Matrizen über \(K\) sind isomorph.
    \cite{Satz ~10.2}
    \end{field}
\end{note}

\begin{note}
    \tags{Def}
    \xplain{51387522-ba4a-11ec-8422-0242ac120002}
    \begin{field}
    Die \textbf{darstellende Matrix} einer Bilinearform \(\beta\) bezüglich einer Basis \(B\) ist gegeben durch \hide{\[M_B(\beta)\colon = \beta(\vec{b}_i, \vec{b}_j)_{ij}\]}
    \end{field}
    \begin{field}
    Die \textbf{darstellende Matrix} einer Bilinearform \(\beta\) bezüglich einer Basis \(B\) ist gegeben durch
    \[M_B(\beta)\colon = \beta(\vec{b}_i, \vec{b}_j)_{ij}\]
    \cite{Def. ~10.3}
    \end{field}
\end{note}

\begin{note}
    \tags{Def}
    \xplain{b9664264-bb12-11ec-8422-0242ac120002}
    \begin{field}
        Zwei Matrizen \(A, A'\) sind \textbf{kongruent}, falls es \hide{eine invertierbare Matrix \(S\) gibt mit}
    \end{field}

    \begin{field}
        Zwei quadratische Matrizen \(A, A'\) sind \textbf{kongruent}, falls es eine invertierbare Matrix \(S\) gibt mit
        \[ A' = S^TAS\]
    \cite{Def. ~10.5}
    \end{field}
\end{note}

%%%%%%%%% Vorlesung 2 %%%%%%%%%%%%
\end{document}
